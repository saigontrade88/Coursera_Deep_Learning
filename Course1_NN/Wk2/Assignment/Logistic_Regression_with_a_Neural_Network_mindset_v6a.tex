
% Default to the notebook output style

    


% Inherit from the specified cell style.




    
\documentclass[11pt]{article}

    
    
    \usepackage[T1]{fontenc}
    % Nicer default font than Computer Modern for most use cases
    \usepackage{palatino}

    % Basic figure setup, for now with no caption control since it's done
    % automatically by Pandoc (which extracts ![](path) syntax from Markdown).
    \usepackage{graphicx}
    % We will generate all images so they have a width \maxwidth. This means
    % that they will get their normal width if they fit onto the page, but
    % are scaled down if they would overflow the margins.
    \makeatletter
    \def\maxwidth{\ifdim\Gin@nat@width>\linewidth\linewidth
    \else\Gin@nat@width\fi}
    \makeatother
    \let\Oldincludegraphics\includegraphics
    % Set max figure width to be 80% of text width, for now hardcoded.
    \renewcommand{\includegraphics}[1]{\Oldincludegraphics[width=.8\maxwidth]{#1}}
    % Ensure that by default, figures have no caption (until we provide a
    % proper Figure object with a Caption API and a way to capture that
    % in the conversion process - todo).
    \usepackage{caption}
    \DeclareCaptionLabelFormat{nolabel}{}
    \captionsetup{labelformat=nolabel}

    \usepackage{adjustbox} % Used to constrain images to a maximum size 
    \usepackage{xcolor} % Allow colors to be defined
    \usepackage{enumerate} % Needed for markdown enumerations to work
    \usepackage{geometry} % Used to adjust the document margins
    \usepackage{amsmath} % Equations
    \usepackage{amssymb} % Equations
    \usepackage{textcomp} % defines textquotesingle
    % Hack from http://tex.stackexchange.com/a/47451/13684:
    \AtBeginDocument{%
        \def\PYZsq{\textquotesingle}% Upright quotes in Pygmentized code
    }
    \usepackage{upquote} % Upright quotes for verbatim code
    \usepackage{eurosym} % defines \euro
    \usepackage[mathletters]{ucs} % Extended unicode (utf-8) support
    \usepackage[utf8x]{inputenc} % Allow utf-8 characters in the tex document
    \usepackage{fancyvrb} % verbatim replacement that allows latex
    \usepackage{grffile} % extends the file name processing of package graphics 
                         % to support a larger range 
    % The hyperref package gives us a pdf with properly built
    % internal navigation ('pdf bookmarks' for the table of contents,
    % internal cross-reference links, web links for URLs, etc.)
    \usepackage{hyperref}
    \usepackage{longtable} % longtable support required by pandoc >1.10
    \usepackage{booktabs}  % table support for pandoc > 1.12.2
    \usepackage[normalem]{ulem} % ulem is needed to support strikethroughs (\sout)
                                % normalem makes italics be italics, not underlines
    

    
    
    % Colors for the hyperref package
    \definecolor{urlcolor}{rgb}{0,.145,.698}
    \definecolor{linkcolor}{rgb}{.71,0.21,0.01}
    \definecolor{citecolor}{rgb}{.12,.54,.11}

    % ANSI colors
    \definecolor{ansi-black}{HTML}{3E424D}
    \definecolor{ansi-black-intense}{HTML}{282C36}
    \definecolor{ansi-red}{HTML}{E75C58}
    \definecolor{ansi-red-intense}{HTML}{B22B31}
    \definecolor{ansi-green}{HTML}{00A250}
    \definecolor{ansi-green-intense}{HTML}{007427}
    \definecolor{ansi-yellow}{HTML}{DDB62B}
    \definecolor{ansi-yellow-intense}{HTML}{B27D12}
    \definecolor{ansi-blue}{HTML}{208FFB}
    \definecolor{ansi-blue-intense}{HTML}{0065CA}
    \definecolor{ansi-magenta}{HTML}{D160C4}
    \definecolor{ansi-magenta-intense}{HTML}{A03196}
    \definecolor{ansi-cyan}{HTML}{60C6C8}
    \definecolor{ansi-cyan-intense}{HTML}{258F8F}
    \definecolor{ansi-white}{HTML}{C5C1B4}
    \definecolor{ansi-white-intense}{HTML}{A1A6B2}

    % commands and environments needed by pandoc snippets
    % extracted from the output of `pandoc -s`
    \providecommand{\tightlist}{%
      \setlength{\itemsep}{0pt}\setlength{\parskip}{0pt}}
    \DefineVerbatimEnvironment{Highlighting}{Verbatim}{commandchars=\\\{\}}
    % Add ',fontsize=\small' for more characters per line
    \newenvironment{Shaded}{}{}
    \newcommand{\KeywordTok}[1]{\textcolor[rgb]{0.00,0.44,0.13}{\textbf{{#1}}}}
    \newcommand{\DataTypeTok}[1]{\textcolor[rgb]{0.56,0.13,0.00}{{#1}}}
    \newcommand{\DecValTok}[1]{\textcolor[rgb]{0.25,0.63,0.44}{{#1}}}
    \newcommand{\BaseNTok}[1]{\textcolor[rgb]{0.25,0.63,0.44}{{#1}}}
    \newcommand{\FloatTok}[1]{\textcolor[rgb]{0.25,0.63,0.44}{{#1}}}
    \newcommand{\CharTok}[1]{\textcolor[rgb]{0.25,0.44,0.63}{{#1}}}
    \newcommand{\StringTok}[1]{\textcolor[rgb]{0.25,0.44,0.63}{{#1}}}
    \newcommand{\CommentTok}[1]{\textcolor[rgb]{0.38,0.63,0.69}{\textit{{#1}}}}
    \newcommand{\OtherTok}[1]{\textcolor[rgb]{0.00,0.44,0.13}{{#1}}}
    \newcommand{\AlertTok}[1]{\textcolor[rgb]{1.00,0.00,0.00}{\textbf{{#1}}}}
    \newcommand{\FunctionTok}[1]{\textcolor[rgb]{0.02,0.16,0.49}{{#1}}}
    \newcommand{\RegionMarkerTok}[1]{{#1}}
    \newcommand{\ErrorTok}[1]{\textcolor[rgb]{1.00,0.00,0.00}{\textbf{{#1}}}}
    \newcommand{\NormalTok}[1]{{#1}}
    
    % Additional commands for more recent versions of Pandoc
    \newcommand{\ConstantTok}[1]{\textcolor[rgb]{0.53,0.00,0.00}{{#1}}}
    \newcommand{\SpecialCharTok}[1]{\textcolor[rgb]{0.25,0.44,0.63}{{#1}}}
    \newcommand{\VerbatimStringTok}[1]{\textcolor[rgb]{0.25,0.44,0.63}{{#1}}}
    \newcommand{\SpecialStringTok}[1]{\textcolor[rgb]{0.73,0.40,0.53}{{#1}}}
    \newcommand{\ImportTok}[1]{{#1}}
    \newcommand{\DocumentationTok}[1]{\textcolor[rgb]{0.73,0.13,0.13}{\textit{{#1}}}}
    \newcommand{\AnnotationTok}[1]{\textcolor[rgb]{0.38,0.63,0.69}{\textbf{\textit{{#1}}}}}
    \newcommand{\CommentVarTok}[1]{\textcolor[rgb]{0.38,0.63,0.69}{\textbf{\textit{{#1}}}}}
    \newcommand{\VariableTok}[1]{\textcolor[rgb]{0.10,0.09,0.49}{{#1}}}
    \newcommand{\ControlFlowTok}[1]{\textcolor[rgb]{0.00,0.44,0.13}{\textbf{{#1}}}}
    \newcommand{\OperatorTok}[1]{\textcolor[rgb]{0.40,0.40,0.40}{{#1}}}
    \newcommand{\BuiltInTok}[1]{{#1}}
    \newcommand{\ExtensionTok}[1]{{#1}}
    \newcommand{\PreprocessorTok}[1]{\textcolor[rgb]{0.74,0.48,0.00}{{#1}}}
    \newcommand{\AttributeTok}[1]{\textcolor[rgb]{0.49,0.56,0.16}{{#1}}}
    \newcommand{\InformationTok}[1]{\textcolor[rgb]{0.38,0.63,0.69}{\textbf{\textit{{#1}}}}}
    \newcommand{\WarningTok}[1]{\textcolor[rgb]{0.38,0.63,0.69}{\textbf{\textit{{#1}}}}}
    
    
    % Define a nice break command that doesn't care if a line doesn't already
    % exist.
    \def\br{\hspace*{\fill} \\* }
    % Math Jax compatability definitions
    \def\gt{>}
    \def\lt{<}
    % Document parameters
    \title{Logistic\_Regression\_with\_a\_Neural\_Network\_mindset\_v6a}
    
    
    

    % Pygments definitions
    
\makeatletter
\def\PY@reset{\let\PY@it=\relax \let\PY@bf=\relax%
    \let\PY@ul=\relax \let\PY@tc=\relax%
    \let\PY@bc=\relax \let\PY@ff=\relax}
\def\PY@tok#1{\csname PY@tok@#1\endcsname}
\def\PY@toks#1+{\ifx\relax#1\empty\else%
    \PY@tok{#1}\expandafter\PY@toks\fi}
\def\PY@do#1{\PY@bc{\PY@tc{\PY@ul{%
    \PY@it{\PY@bf{\PY@ff{#1}}}}}}}
\def\PY#1#2{\PY@reset\PY@toks#1+\relax+\PY@do{#2}}

\expandafter\def\csname PY@tok@w\endcsname{\def\PY@tc##1{\textcolor[rgb]{0.73,0.73,0.73}{##1}}}
\expandafter\def\csname PY@tok@c\endcsname{\let\PY@it=\textit\def\PY@tc##1{\textcolor[rgb]{0.25,0.50,0.50}{##1}}}
\expandafter\def\csname PY@tok@cp\endcsname{\def\PY@tc##1{\textcolor[rgb]{0.74,0.48,0.00}{##1}}}
\expandafter\def\csname PY@tok@k\endcsname{\let\PY@bf=\textbf\def\PY@tc##1{\textcolor[rgb]{0.00,0.50,0.00}{##1}}}
\expandafter\def\csname PY@tok@kp\endcsname{\def\PY@tc##1{\textcolor[rgb]{0.00,0.50,0.00}{##1}}}
\expandafter\def\csname PY@tok@kt\endcsname{\def\PY@tc##1{\textcolor[rgb]{0.69,0.00,0.25}{##1}}}
\expandafter\def\csname PY@tok@o\endcsname{\def\PY@tc##1{\textcolor[rgb]{0.40,0.40,0.40}{##1}}}
\expandafter\def\csname PY@tok@ow\endcsname{\let\PY@bf=\textbf\def\PY@tc##1{\textcolor[rgb]{0.67,0.13,1.00}{##1}}}
\expandafter\def\csname PY@tok@nb\endcsname{\def\PY@tc##1{\textcolor[rgb]{0.00,0.50,0.00}{##1}}}
\expandafter\def\csname PY@tok@nf\endcsname{\def\PY@tc##1{\textcolor[rgb]{0.00,0.00,1.00}{##1}}}
\expandafter\def\csname PY@tok@nc\endcsname{\let\PY@bf=\textbf\def\PY@tc##1{\textcolor[rgb]{0.00,0.00,1.00}{##1}}}
\expandafter\def\csname PY@tok@nn\endcsname{\let\PY@bf=\textbf\def\PY@tc##1{\textcolor[rgb]{0.00,0.00,1.00}{##1}}}
\expandafter\def\csname PY@tok@ne\endcsname{\let\PY@bf=\textbf\def\PY@tc##1{\textcolor[rgb]{0.82,0.25,0.23}{##1}}}
\expandafter\def\csname PY@tok@nv\endcsname{\def\PY@tc##1{\textcolor[rgb]{0.10,0.09,0.49}{##1}}}
\expandafter\def\csname PY@tok@no\endcsname{\def\PY@tc##1{\textcolor[rgb]{0.53,0.00,0.00}{##1}}}
\expandafter\def\csname PY@tok@nl\endcsname{\def\PY@tc##1{\textcolor[rgb]{0.63,0.63,0.00}{##1}}}
\expandafter\def\csname PY@tok@ni\endcsname{\let\PY@bf=\textbf\def\PY@tc##1{\textcolor[rgb]{0.60,0.60,0.60}{##1}}}
\expandafter\def\csname PY@tok@na\endcsname{\def\PY@tc##1{\textcolor[rgb]{0.49,0.56,0.16}{##1}}}
\expandafter\def\csname PY@tok@nt\endcsname{\let\PY@bf=\textbf\def\PY@tc##1{\textcolor[rgb]{0.00,0.50,0.00}{##1}}}
\expandafter\def\csname PY@tok@nd\endcsname{\def\PY@tc##1{\textcolor[rgb]{0.67,0.13,1.00}{##1}}}
\expandafter\def\csname PY@tok@s\endcsname{\def\PY@tc##1{\textcolor[rgb]{0.73,0.13,0.13}{##1}}}
\expandafter\def\csname PY@tok@sd\endcsname{\let\PY@it=\textit\def\PY@tc##1{\textcolor[rgb]{0.73,0.13,0.13}{##1}}}
\expandafter\def\csname PY@tok@si\endcsname{\let\PY@bf=\textbf\def\PY@tc##1{\textcolor[rgb]{0.73,0.40,0.53}{##1}}}
\expandafter\def\csname PY@tok@se\endcsname{\let\PY@bf=\textbf\def\PY@tc##1{\textcolor[rgb]{0.73,0.40,0.13}{##1}}}
\expandafter\def\csname PY@tok@sr\endcsname{\def\PY@tc##1{\textcolor[rgb]{0.73,0.40,0.53}{##1}}}
\expandafter\def\csname PY@tok@ss\endcsname{\def\PY@tc##1{\textcolor[rgb]{0.10,0.09,0.49}{##1}}}
\expandafter\def\csname PY@tok@sx\endcsname{\def\PY@tc##1{\textcolor[rgb]{0.00,0.50,0.00}{##1}}}
\expandafter\def\csname PY@tok@m\endcsname{\def\PY@tc##1{\textcolor[rgb]{0.40,0.40,0.40}{##1}}}
\expandafter\def\csname PY@tok@gh\endcsname{\let\PY@bf=\textbf\def\PY@tc##1{\textcolor[rgb]{0.00,0.00,0.50}{##1}}}
\expandafter\def\csname PY@tok@gu\endcsname{\let\PY@bf=\textbf\def\PY@tc##1{\textcolor[rgb]{0.50,0.00,0.50}{##1}}}
\expandafter\def\csname PY@tok@gd\endcsname{\def\PY@tc##1{\textcolor[rgb]{0.63,0.00,0.00}{##1}}}
\expandafter\def\csname PY@tok@gi\endcsname{\def\PY@tc##1{\textcolor[rgb]{0.00,0.63,0.00}{##1}}}
\expandafter\def\csname PY@tok@gr\endcsname{\def\PY@tc##1{\textcolor[rgb]{1.00,0.00,0.00}{##1}}}
\expandafter\def\csname PY@tok@ge\endcsname{\let\PY@it=\textit}
\expandafter\def\csname PY@tok@gs\endcsname{\let\PY@bf=\textbf}
\expandafter\def\csname PY@tok@gp\endcsname{\let\PY@bf=\textbf\def\PY@tc##1{\textcolor[rgb]{0.00,0.00,0.50}{##1}}}
\expandafter\def\csname PY@tok@go\endcsname{\def\PY@tc##1{\textcolor[rgb]{0.53,0.53,0.53}{##1}}}
\expandafter\def\csname PY@tok@gt\endcsname{\def\PY@tc##1{\textcolor[rgb]{0.00,0.27,0.87}{##1}}}
\expandafter\def\csname PY@tok@err\endcsname{\def\PY@bc##1{\setlength{\fboxsep}{0pt}\fcolorbox[rgb]{1.00,0.00,0.00}{1,1,1}{\strut ##1}}}
\expandafter\def\csname PY@tok@kc\endcsname{\let\PY@bf=\textbf\def\PY@tc##1{\textcolor[rgb]{0.00,0.50,0.00}{##1}}}
\expandafter\def\csname PY@tok@kd\endcsname{\let\PY@bf=\textbf\def\PY@tc##1{\textcolor[rgb]{0.00,0.50,0.00}{##1}}}
\expandafter\def\csname PY@tok@kn\endcsname{\let\PY@bf=\textbf\def\PY@tc##1{\textcolor[rgb]{0.00,0.50,0.00}{##1}}}
\expandafter\def\csname PY@tok@kr\endcsname{\let\PY@bf=\textbf\def\PY@tc##1{\textcolor[rgb]{0.00,0.50,0.00}{##1}}}
\expandafter\def\csname PY@tok@bp\endcsname{\def\PY@tc##1{\textcolor[rgb]{0.00,0.50,0.00}{##1}}}
\expandafter\def\csname PY@tok@vc\endcsname{\def\PY@tc##1{\textcolor[rgb]{0.10,0.09,0.49}{##1}}}
\expandafter\def\csname PY@tok@vg\endcsname{\def\PY@tc##1{\textcolor[rgb]{0.10,0.09,0.49}{##1}}}
\expandafter\def\csname PY@tok@vi\endcsname{\def\PY@tc##1{\textcolor[rgb]{0.10,0.09,0.49}{##1}}}
\expandafter\def\csname PY@tok@sb\endcsname{\def\PY@tc##1{\textcolor[rgb]{0.73,0.13,0.13}{##1}}}
\expandafter\def\csname PY@tok@sc\endcsname{\def\PY@tc##1{\textcolor[rgb]{0.73,0.13,0.13}{##1}}}
\expandafter\def\csname PY@tok@s2\endcsname{\def\PY@tc##1{\textcolor[rgb]{0.73,0.13,0.13}{##1}}}
\expandafter\def\csname PY@tok@sh\endcsname{\def\PY@tc##1{\textcolor[rgb]{0.73,0.13,0.13}{##1}}}
\expandafter\def\csname PY@tok@s1\endcsname{\def\PY@tc##1{\textcolor[rgb]{0.73,0.13,0.13}{##1}}}
\expandafter\def\csname PY@tok@mb\endcsname{\def\PY@tc##1{\textcolor[rgb]{0.40,0.40,0.40}{##1}}}
\expandafter\def\csname PY@tok@mf\endcsname{\def\PY@tc##1{\textcolor[rgb]{0.40,0.40,0.40}{##1}}}
\expandafter\def\csname PY@tok@mh\endcsname{\def\PY@tc##1{\textcolor[rgb]{0.40,0.40,0.40}{##1}}}
\expandafter\def\csname PY@tok@mi\endcsname{\def\PY@tc##1{\textcolor[rgb]{0.40,0.40,0.40}{##1}}}
\expandafter\def\csname PY@tok@il\endcsname{\def\PY@tc##1{\textcolor[rgb]{0.40,0.40,0.40}{##1}}}
\expandafter\def\csname PY@tok@mo\endcsname{\def\PY@tc##1{\textcolor[rgb]{0.40,0.40,0.40}{##1}}}
\expandafter\def\csname PY@tok@ch\endcsname{\let\PY@it=\textit\def\PY@tc##1{\textcolor[rgb]{0.25,0.50,0.50}{##1}}}
\expandafter\def\csname PY@tok@cm\endcsname{\let\PY@it=\textit\def\PY@tc##1{\textcolor[rgb]{0.25,0.50,0.50}{##1}}}
\expandafter\def\csname PY@tok@cpf\endcsname{\let\PY@it=\textit\def\PY@tc##1{\textcolor[rgb]{0.25,0.50,0.50}{##1}}}
\expandafter\def\csname PY@tok@c1\endcsname{\let\PY@it=\textit\def\PY@tc##1{\textcolor[rgb]{0.25,0.50,0.50}{##1}}}
\expandafter\def\csname PY@tok@cs\endcsname{\let\PY@it=\textit\def\PY@tc##1{\textcolor[rgb]{0.25,0.50,0.50}{##1}}}

\def\PYZbs{\char`\\}
\def\PYZus{\char`\_}
\def\PYZob{\char`\{}
\def\PYZcb{\char`\}}
\def\PYZca{\char`\^}
\def\PYZam{\char`\&}
\def\PYZlt{\char`\<}
\def\PYZgt{\char`\>}
\def\PYZsh{\char`\#}
\def\PYZpc{\char`\%}
\def\PYZdl{\char`\$}
\def\PYZhy{\char`\-}
\def\PYZsq{\char`\'}
\def\PYZdq{\char`\"}
\def\PYZti{\char`\~}
% for compatibility with earlier versions
\def\PYZat{@}
\def\PYZlb{[}
\def\PYZrb{]}
\makeatother


    % Exact colors from NB
    \definecolor{incolor}{rgb}{0.0, 0.0, 0.5}
    \definecolor{outcolor}{rgb}{0.545, 0.0, 0.0}



    
    % Prevent overflowing lines due to hard-to-break entities
    \sloppy 
    % Setup hyperref package
    \hypersetup{
      breaklinks=true,  % so long urls are correctly broken across lines
      colorlinks=true,
      urlcolor=urlcolor,
      linkcolor=linkcolor,
      citecolor=citecolor,
      }
    % Slightly bigger margins than the latex defaults
    
    \geometry{verbose,tmargin=1in,bmargin=1in,lmargin=1in,rmargin=1in}
    
    

    \begin{document}
    
    
    \maketitle
    
    

    
    \section{Logistic Regression with a Neural Network
mindset}\label{logistic-regression-with-a-neural-network-mindset}

Welcome to your first (required) programming assignment! You will build
a logistic regression classifier to recognize cats. This assignment will
step you through how to do this with a Neural Network mindset, and so
will also hone your intuitions about deep learning.

\textbf{Instructions:} - Do not use loops (for/while) in your code,
unless the instructions explicitly ask you to do so.

\textbf{You will learn to:} - Build the general architecture of a
learning algorithm, including: - Initializing parameters - Calculating
the cost function and its gradient - Using an optimization algorithm
(gradient descent) - Gather all three functions above into a main model
function, in the right order.

    \subsection{Updates}\label{updates}

This notebook has been updated over the past few months. The prior
version was named ``v5'', and the current versionis now named `6a'

\paragraph{If you were working on a previous
version:}\label{if-you-were-working-on-a-previous-version}

\begin{itemize}
\itemsep1pt\parskip0pt\parsep0pt
\item
  You can find your prior work by looking in the file directory for the
  older files (named by version name).
\item
  To view the file directory, click on the ``Coursera'' icon in the top
  left corner of this notebook.
\item
  Please copy your work from the older versions to the new version, in
  order to submit your work for grading.
\end{itemize}

\paragraph{List of Updates}\label{list-of-updates}

\begin{itemize}
\itemsep1pt\parskip0pt\parsep0pt
\item
  Forward propagation formula, indexing now starts at 1 instead of 0.
\item
  Optimization function comment now says ``print cost every 100 training
  iterations'' instead of ``examples''.
\item
  Fixed grammar in the comments.
\item
  Y\_prediction\_test variable name is used consistently.
\item
  Plot's axis label now says ``iterations (hundred)'' instead of
  ``iterations''.
\item
  When testing the model, the test image is normalized by dividing by
  255.
\end{itemize}

    \subsection{1 - Packages}\label{packages}

First, let's run the cell below to import all the packages that you will
need during this assignment. - \href{www.numpy.org}{numpy} is the
fundamental package for scientific computing with Python. -
\href{http://www.h5py.org}{h5py} is a common package to interact with a
dataset that is stored on an H5 file. -
\href{http://matplotlib.org}{matplotlib} is a famous library to plot
graphs in Python. - \href{http://www.pythonware.com/products/pil/}{PIL}
and \href{https://www.scipy.org/}{scipy} are used here to test your
model with your own picture at the end.

    \begin{Verbatim}[commandchars=\\\{\}]
{\color{incolor}In [{\color{incolor} }]:} \PY{k+kn}{import} \PY{n+nn}{numpy} \PY{k}{as} \PY{n+nn}{np}
        \PY{k+kn}{import} \PY{n+nn}{matplotlib}\PY{n+nn}{.}\PY{n+nn}{pyplot} \PY{k}{as} \PY{n+nn}{plt}
        \PY{k+kn}{import} \PY{n+nn}{h5py}
        \PY{k+kn}{import} \PY{n+nn}{scipy}
        \PY{k+kn}{from} \PY{n+nn}{PIL} \PY{k}{import} \PY{n}{Image}
        \PY{k+kn}{from} \PY{n+nn}{scipy} \PY{k}{import} \PY{n}{ndimage}
        \PY{k+kn}{from} \PY{n+nn}{lr\PYZus{}utils} \PY{k}{import} \PY{n}{load\PYZus{}dataset}
        
        \PY{o}{\PYZpc{}}\PY{k}{matplotlib} inline
\end{Verbatim}

    \subsection{2 - Overview of the Problem
set}\label{overview-of-the-problem-set}

\textbf{Problem Statement}: You are given a dataset (``data.h5'')
containing: - a training set of m\_train images labeled as cat (y=1) or
non-cat (y=0) - a test set of m\_test images labeled as cat or non-cat -
each image is of shape (num\_px, num\_px, 3) where 3 is for the 3
channels (RGB). Thus, each image is square (height = num\_px) and (width
= num\_px).

You will build a simple image-recognition algorithm that can correctly
classify pictures as cat or non-cat.

Let's get more familiar with the dataset. Load the data by running the
following code.

    \begin{Verbatim}[commandchars=\\\{\}]
{\color{incolor}In [{\color{incolor} }]:} \PY{c+c1}{\PYZsh{} Loading the data (cat/non\PYZhy{}cat)}
        \PY{n}{train\PYZus{}set\PYZus{}x\PYZus{}orig}\PY{p}{,} \PY{n}{train\PYZus{}set\PYZus{}y}\PY{p}{,} \PY{n}{test\PYZus{}set\PYZus{}x\PYZus{}orig}\PY{p}{,} \PY{n}{test\PYZus{}set\PYZus{}y}\PY{p}{,} \PY{n}{classes} \PY{o}{=} \PY{n}{load\PYZus{}dataset}\PY{p}{(}\PY{p}{)}
\end{Verbatim}

    We added ``\emph{orig" at the end of image datasets (train and test)
because we are going to preprocess them. After preprocessing, we will
end up with train}set\_x and test\_set\_x (the labels train\_set\_y and
test\_set\_y don't need any preprocessing).

Each line of your train\_set\_x\_orig and test\_set\_x\_orig is an array
representing an image. You can visualize an example by running the
following code. Feel free also to change the \texttt{index} value and
re-run to see other images.

    \begin{Verbatim}[commandchars=\\\{\}]
{\color{incolor}In [{\color{incolor} }]:} \PY{c+c1}{\PYZsh{} Example of a picture}
        \PY{n}{index} \PY{o}{=} \PY{l+m+mi}{25}
        \PY{n}{plt}\PY{o}{.}\PY{n}{imshow}\PY{p}{(}\PY{n}{train\PYZus{}set\PYZus{}x\PYZus{}orig}\PY{p}{[}\PY{n}{index}\PY{p}{]}\PY{p}{)}
        \PY{n+nb}{print} \PY{p}{(}\PY{l+s+s2}{\PYZdq{}}\PY{l+s+s2}{y = }\PY{l+s+s2}{\PYZdq{}} \PY{o}{+} \PY{n+nb}{str}\PY{p}{(}\PY{n}{train\PYZus{}set\PYZus{}y}\PY{p}{[}\PY{p}{:}\PY{p}{,} \PY{n}{index}\PY{p}{]}\PY{p}{)} \PY{o}{+} \PY{l+s+s2}{\PYZdq{}}\PY{l+s+s2}{, it}\PY{l+s+s2}{\PYZsq{}}\PY{l+s+s2}{s a }\PY{l+s+s2}{\PYZsq{}}\PY{l+s+s2}{\PYZdq{}} \PY{o}{+} \PY{n}{classes}\PY{p}{[}\PY{n}{np}\PY{o}{.}\PY{n}{squeeze}\PY{p}{(}\PY{n}{train\PYZus{}set\PYZus{}y}\PY{p}{[}\PY{p}{:}\PY{p}{,} \PY{n}{index}\PY{p}{]}\PY{p}{)}\PY{p}{]}\PY{o}{.}\PY{n}{decode}\PY{p}{(}\PY{l+s+s2}{\PYZdq{}}\PY{l+s+s2}{utf\PYZhy{}8}\PY{l+s+s2}{\PYZdq{}}\PY{p}{)} \PY{o}{+}  \PY{l+s+s2}{\PYZdq{}}\PY{l+s+s2}{\PYZsq{}}\PY{l+s+s2}{ picture.}\PY{l+s+s2}{\PYZdq{}}\PY{p}{)}
\end{Verbatim}

    Many software bugs in deep learning come from having matrix/vector
dimensions that don't fit. If you can keep your matrix/vector dimensions
straight you will go a long way toward eliminating many bugs.

\textbf{Exercise:} Find the values for: - m\_train (number of training
examples) - m\_test (number of test examples) - num\_px (= height =
width of a training image) Remember that \texttt{train\_set\_x\_orig} is
a numpy-array of shape (m\_train, num\_px, num\_px, 3). For instance,
you can access \texttt{m\_train} by writing
\texttt{train\_set\_x\_orig.shape{[}0{]}}.

    \begin{Verbatim}[commandchars=\\\{\}]
{\color{incolor}In [{\color{incolor} }]:} \PY{c+c1}{\PYZsh{}\PYZsh{}\PYZsh{} START CODE HERE \PYZsh{}\PYZsh{}\PYZsh{} (≈ 3 lines of code)}
        \PY{n}{m\PYZus{}train} \PY{o}{=} \PY{k+kc}{None}
        \PY{n}{m\PYZus{}test} \PY{o}{=} \PY{k+kc}{None}
        \PY{n}{num\PYZus{}px} \PY{o}{=} \PY{k+kc}{None}
        \PY{c+c1}{\PYZsh{}\PYZsh{}\PYZsh{} END CODE HERE \PYZsh{}\PYZsh{}\PYZsh{}}
        
        \PY{n+nb}{print} \PY{p}{(}\PY{l+s+s2}{\PYZdq{}}\PY{l+s+s2}{Number of training examples: m\PYZus{}train = }\PY{l+s+s2}{\PYZdq{}} \PY{o}{+} \PY{n+nb}{str}\PY{p}{(}\PY{n}{m\PYZus{}train}\PY{p}{)}\PY{p}{)}
        \PY{n+nb}{print} \PY{p}{(}\PY{l+s+s2}{\PYZdq{}}\PY{l+s+s2}{Number of testing examples: m\PYZus{}test = }\PY{l+s+s2}{\PYZdq{}} \PY{o}{+} \PY{n+nb}{str}\PY{p}{(}\PY{n}{m\PYZus{}test}\PY{p}{)}\PY{p}{)}
        \PY{n+nb}{print} \PY{p}{(}\PY{l+s+s2}{\PYZdq{}}\PY{l+s+s2}{Height/Width of each image: num\PYZus{}px = }\PY{l+s+s2}{\PYZdq{}} \PY{o}{+} \PY{n+nb}{str}\PY{p}{(}\PY{n}{num\PYZus{}px}\PY{p}{)}\PY{p}{)}
        \PY{n+nb}{print} \PY{p}{(}\PY{l+s+s2}{\PYZdq{}}\PY{l+s+s2}{Each image is of size: (}\PY{l+s+s2}{\PYZdq{}} \PY{o}{+} \PY{n+nb}{str}\PY{p}{(}\PY{n}{num\PYZus{}px}\PY{p}{)} \PY{o}{+} \PY{l+s+s2}{\PYZdq{}}\PY{l+s+s2}{, }\PY{l+s+s2}{\PYZdq{}} \PY{o}{+} \PY{n+nb}{str}\PY{p}{(}\PY{n}{num\PYZus{}px}\PY{p}{)} \PY{o}{+} \PY{l+s+s2}{\PYZdq{}}\PY{l+s+s2}{, 3)}\PY{l+s+s2}{\PYZdq{}}\PY{p}{)}
        \PY{n+nb}{print} \PY{p}{(}\PY{l+s+s2}{\PYZdq{}}\PY{l+s+s2}{train\PYZus{}set\PYZus{}x shape: }\PY{l+s+s2}{\PYZdq{}} \PY{o}{+} \PY{n+nb}{str}\PY{p}{(}\PY{n}{train\PYZus{}set\PYZus{}x\PYZus{}orig}\PY{o}{.}\PY{n}{shape}\PY{p}{)}\PY{p}{)}
        \PY{n+nb}{print} \PY{p}{(}\PY{l+s+s2}{\PYZdq{}}\PY{l+s+s2}{train\PYZus{}set\PYZus{}y shape: }\PY{l+s+s2}{\PYZdq{}} \PY{o}{+} \PY{n+nb}{str}\PY{p}{(}\PY{n}{train\PYZus{}set\PYZus{}y}\PY{o}{.}\PY{n}{shape}\PY{p}{)}\PY{p}{)}
        \PY{n+nb}{print} \PY{p}{(}\PY{l+s+s2}{\PYZdq{}}\PY{l+s+s2}{test\PYZus{}set\PYZus{}x shape: }\PY{l+s+s2}{\PYZdq{}} \PY{o}{+} \PY{n+nb}{str}\PY{p}{(}\PY{n}{test\PYZus{}set\PYZus{}x\PYZus{}orig}\PY{o}{.}\PY{n}{shape}\PY{p}{)}\PY{p}{)}
        \PY{n+nb}{print} \PY{p}{(}\PY{l+s+s2}{\PYZdq{}}\PY{l+s+s2}{test\PYZus{}set\PYZus{}y shape: }\PY{l+s+s2}{\PYZdq{}} \PY{o}{+} \PY{n+nb}{str}\PY{p}{(}\PY{n}{test\PYZus{}set\PYZus{}y}\PY{o}{.}\PY{n}{shape}\PY{p}{)}\PY{p}{)}
\end{Verbatim}

    \textbf{Expected Output for m\_train, m\_test and num\_px}:

\textbf{m\_train}

209

\textbf{m\_test}

50

\textbf{num\_px}

64

    For convenience, you should now reshape images of shape (num\_px,
num\_px, 3) in a numpy-array of shape (num\_px $*$ num\_px $*$ 3, 1).
After this, our training (and test) dataset is a numpy-array where each
column represents a flattened image. There should be m\_train
(respectively m\_test) columns.

\textbf{Exercise:} Reshape the training and test data sets so that
images of size (num\_px, num\_px, 3) are flattened into single vectors
of shape (num\_px $*$ num\_px $*$ 3, 1).

A trick when you want to flatten a matrix X of shape (a,b,c,d) to a
matrix X\_flatten of shape (b$*$c$*$d, a) is to use:

\begin{Shaded}
\begin{Highlighting}[]
\NormalTok{X_flatten = X.reshape(X.shape[}\DecValTok{0}\NormalTok{], -}\DecValTok{1}\NormalTok{).T      }\CommentTok{# X.T is the transpose of X}
\end{Highlighting}
\end{Shaded}

    \begin{Verbatim}[commandchars=\\\{\}]
{\color{incolor}In [{\color{incolor} }]:} \PY{c+c1}{\PYZsh{} Reshape the training and test examples}
        
        \PY{c+c1}{\PYZsh{}\PYZsh{}\PYZsh{} START CODE HERE \PYZsh{}\PYZsh{}\PYZsh{} (≈ 2 lines of code)}
        \PY{n}{train\PYZus{}set\PYZus{}x\PYZus{}flatten} \PY{o}{=} \PY{k+kc}{None}
        \PY{n}{test\PYZus{}set\PYZus{}x\PYZus{}flatten} \PY{o}{=} \PY{k+kc}{None}
        \PY{c+c1}{\PYZsh{}\PYZsh{}\PYZsh{} END CODE HERE \PYZsh{}\PYZsh{}\PYZsh{}}
        
        \PY{n+nb}{print} \PY{p}{(}\PY{l+s+s2}{\PYZdq{}}\PY{l+s+s2}{train\PYZus{}set\PYZus{}x\PYZus{}flatten shape: }\PY{l+s+s2}{\PYZdq{}} \PY{o}{+} \PY{n+nb}{str}\PY{p}{(}\PY{n}{train\PYZus{}set\PYZus{}x\PYZus{}flatten}\PY{o}{.}\PY{n}{shape}\PY{p}{)}\PY{p}{)}
        \PY{n+nb}{print} \PY{p}{(}\PY{l+s+s2}{\PYZdq{}}\PY{l+s+s2}{train\PYZus{}set\PYZus{}y shape: }\PY{l+s+s2}{\PYZdq{}} \PY{o}{+} \PY{n+nb}{str}\PY{p}{(}\PY{n}{train\PYZus{}set\PYZus{}y}\PY{o}{.}\PY{n}{shape}\PY{p}{)}\PY{p}{)}
        \PY{n+nb}{print} \PY{p}{(}\PY{l+s+s2}{\PYZdq{}}\PY{l+s+s2}{test\PYZus{}set\PYZus{}x\PYZus{}flatten shape: }\PY{l+s+s2}{\PYZdq{}} \PY{o}{+} \PY{n+nb}{str}\PY{p}{(}\PY{n}{test\PYZus{}set\PYZus{}x\PYZus{}flatten}\PY{o}{.}\PY{n}{shape}\PY{p}{)}\PY{p}{)}
        \PY{n+nb}{print} \PY{p}{(}\PY{l+s+s2}{\PYZdq{}}\PY{l+s+s2}{test\PYZus{}set\PYZus{}y shape: }\PY{l+s+s2}{\PYZdq{}} \PY{o}{+} \PY{n+nb}{str}\PY{p}{(}\PY{n}{test\PYZus{}set\PYZus{}y}\PY{o}{.}\PY{n}{shape}\PY{p}{)}\PY{p}{)}
        \PY{n+nb}{print} \PY{p}{(}\PY{l+s+s2}{\PYZdq{}}\PY{l+s+s2}{sanity check after reshaping: }\PY{l+s+s2}{\PYZdq{}} \PY{o}{+} \PY{n+nb}{str}\PY{p}{(}\PY{n}{train\PYZus{}set\PYZus{}x\PYZus{}flatten}\PY{p}{[}\PY{l+m+mi}{0}\PY{p}{:}\PY{l+m+mi}{5}\PY{p}{,}\PY{l+m+mi}{0}\PY{p}{]}\PY{p}{)}\PY{p}{)}
\end{Verbatim}

    \textbf{Expected Output}:

\textbf{train\_set\_x\_flatten shape}

(12288, 209)

\textbf{train\_set\_y shape}

(1, 209)

\textbf{test\_set\_x\_flatten shape}

(12288, 50)

\textbf{test\_set\_y shape}

(1, 50)

\textbf{sanity check after reshaping}

{[}17 31 56 22 33{]}

    To represent color images, the red, green and blue channels (RGB) must
be specified for each pixel, and so the pixel value is actually a vector
of three numbers ranging from 0 to 255.

One common preprocessing step in machine learning is to center and
standardize your dataset, meaning that you substract the mean of the
whole numpy array from each example, and then divide each example by the
standard deviation of the whole numpy array. But for picture datasets,
it is simpler and more convenient and works almost as well to just
divide every row of the dataset by 255 (the maximum value of a pixel
channel).

Let's standardize our dataset.

    \begin{Verbatim}[commandchars=\\\{\}]
{\color{incolor}In [{\color{incolor} }]:} \PY{n}{train\PYZus{}set\PYZus{}x} \PY{o}{=} \PY{n}{train\PYZus{}set\PYZus{}x\PYZus{}flatten}\PY{o}{/}\PY{l+m+mf}{255.}
        \PY{n}{test\PYZus{}set\PYZus{}x} \PY{o}{=} \PY{n}{test\PYZus{}set\PYZus{}x\PYZus{}flatten}\PY{o}{/}\PY{l+m+mf}{255.}
\end{Verbatim}

     \textbf{What you need to remember:}

Common steps for pre-processing a new dataset are: - Figure out the
dimensions and shapes of the problem (m\_train, m\_test, num\_px,
\ldots{}) - Reshape the datasets such that each example is now a vector
of size (num\_px * num\_px * 3, 1) - ``Standardize'' the data

    \subsection{3 - General Architecture of the learning
algorithm}\label{general-architecture-of-the-learning-algorithm}

It's time to design a simple algorithm to distinguish cat images from
non-cat images.

You will build a Logistic Regression, using a Neural Network mindset.
The following Figure explains why \textbf{Logistic Regression is
actually a very simple Neural Network!}

\textbf{Mathematical expression of the algorithm}:

For one example $x^{(i)}$: \[z^{(i)} = w^T x^{(i)} + b \tag{1}\]
\[\hat{y}^{(i)} = a^{(i)} = sigmoid(z^{(i)})\tag{2}\]
\[ \mathcal{L}(a^{(i)}, y^{(i)}) =  - y^{(i)}  \log(a^{(i)}) - (1-y^{(i)} )  \log(1-a^{(i)})\tag{3}\]

The cost is then computed by summing over all training examples:
\[ J = \frac{1}{m} \sum_{i=1}^m \mathcal{L}(a^{(i)}, y^{(i)})\tag{6}\]

\textbf{Key steps}: In this exercise, you will carry out the following
steps: - Initialize the parameters of the model - Learn the parameters
for the model by minimizing the cost\\ - Use the learned parameters to
make predictions (on the test set) - Analyse the results and conclude

    \subsection{4 - Building the parts of our
algorithm}\label{building-the-parts-of-our-algorithm}

The main steps for building a Neural Network are: 1. Define the model
structure (such as number of input features) 2. Initialize the model's
parameters 3. Loop: - Calculate current loss (forward propagation) -
Calculate current gradient (backward propagation) - Update parameters
(gradient descent)

You often build 1-3 separately and integrate them into one function we
call \texttt{model()}.

\subsubsection{4.1 - Helper functions}\label{helper-functions}

\textbf{Exercise}: Using your code from ``Python Basics'', implement
\texttt{sigmoid()}. As you've seen in the figure above, you need to
compute $sigmoid( w^T x + b) = \frac{1}{1 + e^{-(w^T x + b)}}$ to make
predictions. Use np.exp().

    \begin{Verbatim}[commandchars=\\\{\}]
{\color{incolor}In [{\color{incolor} }]:} \PY{c+c1}{\PYZsh{} GRADED FUNCTION: sigmoid}
        
        \PY{k}{def} \PY{n+nf}{sigmoid}\PY{p}{(}\PY{n}{z}\PY{p}{)}\PY{p}{:}
            \PY{l+s+sd}{\PYZdq{}\PYZdq{}\PYZdq{}}
        \PY{l+s+sd}{    Compute the sigmoid of z}
        
        \PY{l+s+sd}{    Arguments:}
        \PY{l+s+sd}{    z \PYZhy{}\PYZhy{} A scalar or numpy array of any size.}
        
        \PY{l+s+sd}{    Return:}
        \PY{l+s+sd}{    s \PYZhy{}\PYZhy{} sigmoid(z)}
        \PY{l+s+sd}{    \PYZdq{}\PYZdq{}\PYZdq{}}
        
            \PY{c+c1}{\PYZsh{}\PYZsh{}\PYZsh{} START CODE HERE \PYZsh{}\PYZsh{}\PYZsh{} (≈ 1 line of code)}
            \PY{n}{s} \PY{o}{=} \PY{k+kc}{None}
            \PY{c+c1}{\PYZsh{}\PYZsh{}\PYZsh{} END CODE HERE \PYZsh{}\PYZsh{}\PYZsh{}}
            
            \PY{k}{return} \PY{n}{s}
\end{Verbatim}

    \begin{Verbatim}[commandchars=\\\{\}]
{\color{incolor}In [{\color{incolor} }]:} \PY{n+nb}{print} \PY{p}{(}\PY{l+s+s2}{\PYZdq{}}\PY{l+s+s2}{sigmoid([0, 2]) = }\PY{l+s+s2}{\PYZdq{}} \PY{o}{+} \PY{n+nb}{str}\PY{p}{(}\PY{n}{sigmoid}\PY{p}{(}\PY{n}{np}\PY{o}{.}\PY{n}{array}\PY{p}{(}\PY{p}{[}\PY{l+m+mi}{0}\PY{p}{,}\PY{l+m+mi}{2}\PY{p}{]}\PY{p}{)}\PY{p}{)}\PY{p}{)}\PY{p}{)}
\end{Verbatim}

    \textbf{Expected Output}:

\textbf{sigmoid({[}0, 2{]})}

{[} 0.5 0.88079708{]}

    \subsubsection{4.2 - Initializing
parameters}\label{initializing-parameters}

\textbf{Exercise:} Implement parameter initialization in the cell below.
You have to initialize w as a vector of zeros. If you don't know what
numpy function to use, look up np.zeros() in the Numpy library's
documentation.

    \begin{Verbatim}[commandchars=\\\{\}]
{\color{incolor}In [{\color{incolor} }]:} \PY{c+c1}{\PYZsh{} GRADED FUNCTION: initialize\PYZus{}with\PYZus{}zeros}
        
        \PY{k}{def} \PY{n+nf}{initialize\PYZus{}with\PYZus{}zeros}\PY{p}{(}\PY{n}{dim}\PY{p}{)}\PY{p}{:}
            \PY{l+s+sd}{\PYZdq{}\PYZdq{}\PYZdq{}}
        \PY{l+s+sd}{    This function creates a vector of zeros of shape (dim, 1) for w and initializes b to 0.}
        \PY{l+s+sd}{    }
        \PY{l+s+sd}{    Argument:}
        \PY{l+s+sd}{    dim \PYZhy{}\PYZhy{} size of the w vector we want (or number of parameters in this case)}
        \PY{l+s+sd}{    }
        \PY{l+s+sd}{    Returns:}
        \PY{l+s+sd}{    w \PYZhy{}\PYZhy{} initialized vector of shape (dim, 1)}
        \PY{l+s+sd}{    b \PYZhy{}\PYZhy{} initialized scalar (corresponds to the bias)}
        \PY{l+s+sd}{    \PYZdq{}\PYZdq{}\PYZdq{}}
            
            \PY{c+c1}{\PYZsh{}\PYZsh{}\PYZsh{} START CODE HERE \PYZsh{}\PYZsh{}\PYZsh{} (≈ 1 line of code)}
            \PY{n}{w} \PY{o}{=} \PY{k+kc}{None}
            \PY{n}{b} \PY{o}{=} \PY{k+kc}{None}
            \PY{c+c1}{\PYZsh{}\PYZsh{}\PYZsh{} END CODE HERE \PYZsh{}\PYZsh{}\PYZsh{}}
        
            \PY{k}{assert}\PY{p}{(}\PY{n}{w}\PY{o}{.}\PY{n}{shape} \PY{o}{==} \PY{p}{(}\PY{n}{dim}\PY{p}{,} \PY{l+m+mi}{1}\PY{p}{)}\PY{p}{)}
            \PY{k}{assert}\PY{p}{(}\PY{n+nb}{isinstance}\PY{p}{(}\PY{n}{b}\PY{p}{,} \PY{n+nb}{float}\PY{p}{)} \PY{o+ow}{or} \PY{n+nb}{isinstance}\PY{p}{(}\PY{n}{b}\PY{p}{,} \PY{n+nb}{int}\PY{p}{)}\PY{p}{)}
            
            \PY{k}{return} \PY{n}{w}\PY{p}{,} \PY{n}{b}
\end{Verbatim}

    \begin{Verbatim}[commandchars=\\\{\}]
{\color{incolor}In [{\color{incolor} }]:} \PY{n}{dim} \PY{o}{=} \PY{l+m+mi}{2}
        \PY{n}{w}\PY{p}{,} \PY{n}{b} \PY{o}{=} \PY{n}{initialize\PYZus{}with\PYZus{}zeros}\PY{p}{(}\PY{n}{dim}\PY{p}{)}
        \PY{n+nb}{print} \PY{p}{(}\PY{l+s+s2}{\PYZdq{}}\PY{l+s+s2}{w = }\PY{l+s+s2}{\PYZdq{}} \PY{o}{+} \PY{n+nb}{str}\PY{p}{(}\PY{n}{w}\PY{p}{)}\PY{p}{)}
        \PY{n+nb}{print} \PY{p}{(}\PY{l+s+s2}{\PYZdq{}}\PY{l+s+s2}{b = }\PY{l+s+s2}{\PYZdq{}} \PY{o}{+} \PY{n+nb}{str}\PY{p}{(}\PY{n}{b}\PY{p}{)}\PY{p}{)}
\end{Verbatim}

    \textbf{Expected Output}:

** w **

{[}{[} 0.{]}{[} 0.{]}{]}

** b **

0

For image inputs, w will be of shape (num\_px $\times$ num\_px $\times$
3, 1).

    \subsubsection{4.3 - Forward and Backward
propagation}\label{forward-and-backward-propagation}

Now that your parameters are initialized, you can do the ``forward'' and
``backward'' propagation steps for learning the parameters.

\textbf{Exercise:} Implement a function \texttt{propagate()} that
computes the cost function and its gradient.

\textbf{Hints}:

Forward Propagation: - You get X - You compute
$A = \sigma(w^T X + b) = (a^{(1)}, a^{(2)}, ..., a^{(m-1)}, a^{(m)})$ -
You calculate the cost function:
$J = -\frac{1}{m}\sum_{i=1}^{m}y^{(i)}\log(a^{(i)})+(1-y^{(i)})\log(1-a^{(i)})$

Here are the two formulas you will be using:

\[ \frac{\partial J}{\partial w} = \frac{1}{m}X(A-Y)^T\tag{7}\]
\[ \frac{\partial J}{\partial b} = \frac{1}{m} \sum_{i=1}^m (a^{(i)}-y^{(i)})\tag{8}\]

    \begin{Verbatim}[commandchars=\\\{\}]
{\color{incolor}In [{\color{incolor} }]:} \PY{c+c1}{\PYZsh{} GRADED FUNCTION: propagate}
        
        \PY{k}{def} \PY{n+nf}{propagate}\PY{p}{(}\PY{n}{w}\PY{p}{,} \PY{n}{b}\PY{p}{,} \PY{n}{X}\PY{p}{,} \PY{n}{Y}\PY{p}{)}\PY{p}{:}
            \PY{l+s+sd}{\PYZdq{}\PYZdq{}\PYZdq{}}
        \PY{l+s+sd}{    Implement the cost function and its gradient for the propagation explained above}
        
        \PY{l+s+sd}{    Arguments:}
        \PY{l+s+sd}{    w \PYZhy{}\PYZhy{} weights, a numpy array of size (num\PYZus{}px * num\PYZus{}px * 3, 1)}
        \PY{l+s+sd}{    b \PYZhy{}\PYZhy{} bias, a scalar}
        \PY{l+s+sd}{    X \PYZhy{}\PYZhy{} data of size (num\PYZus{}px * num\PYZus{}px * 3, number of examples)}
        \PY{l+s+sd}{    Y \PYZhy{}\PYZhy{} true \PYZdq{}label\PYZdq{} vector (containing 0 if non\PYZhy{}cat, 1 if cat) of size (1, number of examples)}
        
        \PY{l+s+sd}{    Return:}
        \PY{l+s+sd}{    cost \PYZhy{}\PYZhy{} negative log\PYZhy{}likelihood cost for logistic regression}
        \PY{l+s+sd}{    dw \PYZhy{}\PYZhy{} gradient of the loss with respect to w, thus same shape as w}
        \PY{l+s+sd}{    db \PYZhy{}\PYZhy{} gradient of the loss with respect to b, thus same shape as b}
        \PY{l+s+sd}{    }
        \PY{l+s+sd}{    Tips:}
        \PY{l+s+sd}{    \PYZhy{} Write your code step by step for the propagation. np.log(), np.dot()}
        \PY{l+s+sd}{    \PYZdq{}\PYZdq{}\PYZdq{}}
            
            \PY{n}{m} \PY{o}{=} \PY{n}{X}\PY{o}{.}\PY{n}{shape}\PY{p}{[}\PY{l+m+mi}{1}\PY{p}{]}
            
            \PY{c+c1}{\PYZsh{} FORWARD PROPAGATION (FROM X TO COST)}
            \PY{c+c1}{\PYZsh{}\PYZsh{}\PYZsh{} START CODE HERE \PYZsh{}\PYZsh{}\PYZsh{} (≈ 2 lines of code)}
            \PY{n}{A} \PY{o}{=} \PY{k+kc}{None}                                    \PY{c+c1}{\PYZsh{} compute activation}
            \PY{n}{cost} \PY{o}{=} \PY{k+kc}{None}                                 \PY{c+c1}{\PYZsh{} compute cost}
            \PY{c+c1}{\PYZsh{}\PYZsh{}\PYZsh{} END CODE HERE \PYZsh{}\PYZsh{}\PYZsh{}}
            
            \PY{c+c1}{\PYZsh{} BACKWARD PROPAGATION (TO FIND GRAD)}
            \PY{c+c1}{\PYZsh{}\PYZsh{}\PYZsh{} START CODE HERE \PYZsh{}\PYZsh{}\PYZsh{} (≈ 2 lines of code)}
            \PY{n}{dw} \PY{o}{=} \PY{k+kc}{None}
            \PY{n}{db} \PY{o}{=} \PY{k+kc}{None}
            \PY{c+c1}{\PYZsh{}\PYZsh{}\PYZsh{} END CODE HERE \PYZsh{}\PYZsh{}\PYZsh{}}
        
            \PY{k}{assert}\PY{p}{(}\PY{n}{dw}\PY{o}{.}\PY{n}{shape} \PY{o}{==} \PY{n}{w}\PY{o}{.}\PY{n}{shape}\PY{p}{)}
            \PY{k}{assert}\PY{p}{(}\PY{n}{db}\PY{o}{.}\PY{n}{dtype} \PY{o}{==} \PY{n+nb}{float}\PY{p}{)}
            \PY{n}{cost} \PY{o}{=} \PY{n}{np}\PY{o}{.}\PY{n}{squeeze}\PY{p}{(}\PY{n}{cost}\PY{p}{)}
            \PY{k}{assert}\PY{p}{(}\PY{n}{cost}\PY{o}{.}\PY{n}{shape} \PY{o}{==} \PY{p}{(}\PY{p}{)}\PY{p}{)}
            
            \PY{n}{grads} \PY{o}{=} \PY{p}{\PYZob{}}\PY{l+s+s2}{\PYZdq{}}\PY{l+s+s2}{dw}\PY{l+s+s2}{\PYZdq{}}\PY{p}{:} \PY{n}{dw}\PY{p}{,}
                     \PY{l+s+s2}{\PYZdq{}}\PY{l+s+s2}{db}\PY{l+s+s2}{\PYZdq{}}\PY{p}{:} \PY{n}{db}\PY{p}{\PYZcb{}}
            
            \PY{k}{return} \PY{n}{grads}\PY{p}{,} \PY{n}{cost}
\end{Verbatim}

    \begin{Verbatim}[commandchars=\\\{\}]
{\color{incolor}In [{\color{incolor} }]:} \PY{n}{w}\PY{p}{,} \PY{n}{b}\PY{p}{,} \PY{n}{X}\PY{p}{,} \PY{n}{Y} \PY{o}{=} \PY{n}{np}\PY{o}{.}\PY{n}{array}\PY{p}{(}\PY{p}{[}\PY{p}{[}\PY{l+m+mf}{1.}\PY{p}{]}\PY{p}{,}\PY{p}{[}\PY{l+m+mf}{2.}\PY{p}{]}\PY{p}{]}\PY{p}{)}\PY{p}{,} \PY{l+m+mf}{2.}\PY{p}{,} \PY{n}{np}\PY{o}{.}\PY{n}{array}\PY{p}{(}\PY{p}{[}\PY{p}{[}\PY{l+m+mf}{1.}\PY{p}{,}\PY{l+m+mf}{2.}\PY{p}{,}\PY{o}{\PYZhy{}}\PY{l+m+mf}{1.}\PY{p}{]}\PY{p}{,}\PY{p}{[}\PY{l+m+mf}{3.}\PY{p}{,}\PY{l+m+mf}{4.}\PY{p}{,}\PY{o}{\PYZhy{}}\PY{l+m+mf}{3.2}\PY{p}{]}\PY{p}{]}\PY{p}{)}\PY{p}{,} \PY{n}{np}\PY{o}{.}\PY{n}{array}\PY{p}{(}\PY{p}{[}\PY{p}{[}\PY{l+m+mi}{1}\PY{p}{,}\PY{l+m+mi}{0}\PY{p}{,}\PY{l+m+mi}{1}\PY{p}{]}\PY{p}{]}\PY{p}{)}
        \PY{n}{grads}\PY{p}{,} \PY{n}{cost} \PY{o}{=} \PY{n}{propagate}\PY{p}{(}\PY{n}{w}\PY{p}{,} \PY{n}{b}\PY{p}{,} \PY{n}{X}\PY{p}{,} \PY{n}{Y}\PY{p}{)}
        \PY{n+nb}{print} \PY{p}{(}\PY{l+s+s2}{\PYZdq{}}\PY{l+s+s2}{dw = }\PY{l+s+s2}{\PYZdq{}} \PY{o}{+} \PY{n+nb}{str}\PY{p}{(}\PY{n}{grads}\PY{p}{[}\PY{l+s+s2}{\PYZdq{}}\PY{l+s+s2}{dw}\PY{l+s+s2}{\PYZdq{}}\PY{p}{]}\PY{p}{)}\PY{p}{)}
        \PY{n+nb}{print} \PY{p}{(}\PY{l+s+s2}{\PYZdq{}}\PY{l+s+s2}{db = }\PY{l+s+s2}{\PYZdq{}} \PY{o}{+} \PY{n+nb}{str}\PY{p}{(}\PY{n}{grads}\PY{p}{[}\PY{l+s+s2}{\PYZdq{}}\PY{l+s+s2}{db}\PY{l+s+s2}{\PYZdq{}}\PY{p}{]}\PY{p}{)}\PY{p}{)}
        \PY{n+nb}{print} \PY{p}{(}\PY{l+s+s2}{\PYZdq{}}\PY{l+s+s2}{cost = }\PY{l+s+s2}{\PYZdq{}} \PY{o}{+} \PY{n+nb}{str}\PY{p}{(}\PY{n}{cost}\PY{p}{)}\PY{p}{)}
\end{Verbatim}

    \textbf{Expected Output}:

** dw **

{[}{[} 0.99845601{]}{[} 2.39507239{]}{]}

** db **

0.00145557813678

** cost **

5.801545319394553

    \subsubsection{4.4 - Optimization}\label{optimization}

\begin{itemize}
\itemsep1pt\parskip0pt\parsep0pt
\item
  You have initialized your parameters.
\item
  You are also able to compute a cost function and its gradient.
\item
  Now, you want to update the parameters using gradient descent.
\end{itemize}

\textbf{Exercise:} Write down the optimization function. The goal is to
learn $w$ and $b$ by minimizing the cost function $J$. For a parameter
$\theta$, the update rule is \$ \theta = \theta - \alpha \text{ }
d\theta$, where $\alpha\$ is the learning rate.

    \begin{Verbatim}[commandchars=\\\{\}]
{\color{incolor}In [{\color{incolor} }]:} \PY{c+c1}{\PYZsh{} GRADED FUNCTION: optimize}
        
        \PY{k}{def} \PY{n+nf}{optimize}\PY{p}{(}\PY{n}{w}\PY{p}{,} \PY{n}{b}\PY{p}{,} \PY{n}{X}\PY{p}{,} \PY{n}{Y}\PY{p}{,} \PY{n}{num\PYZus{}iterations}\PY{p}{,} \PY{n}{learning\PYZus{}rate}\PY{p}{,} \PY{n}{print\PYZus{}cost} \PY{o}{=} \PY{k+kc}{False}\PY{p}{)}\PY{p}{:}
            \PY{l+s+sd}{\PYZdq{}\PYZdq{}\PYZdq{}}
        \PY{l+s+sd}{    This function optimizes w and b by running a gradient descent algorithm}
        \PY{l+s+sd}{    }
        \PY{l+s+sd}{    Arguments:}
        \PY{l+s+sd}{    w \PYZhy{}\PYZhy{} weights, a numpy array of size (num\PYZus{}px * num\PYZus{}px * 3, 1)}
        \PY{l+s+sd}{    b \PYZhy{}\PYZhy{} bias, a scalar}
        \PY{l+s+sd}{    X \PYZhy{}\PYZhy{} data of shape (num\PYZus{}px * num\PYZus{}px * 3, number of examples)}
        \PY{l+s+sd}{    Y \PYZhy{}\PYZhy{} true \PYZdq{}label\PYZdq{} vector (containing 0 if non\PYZhy{}cat, 1 if cat), of shape (1, number of examples)}
        \PY{l+s+sd}{    num\PYZus{}iterations \PYZhy{}\PYZhy{} number of iterations of the optimization loop}
        \PY{l+s+sd}{    learning\PYZus{}rate \PYZhy{}\PYZhy{} learning rate of the gradient descent update rule}
        \PY{l+s+sd}{    print\PYZus{}cost \PYZhy{}\PYZhy{} True to print the loss every 100 steps}
        \PY{l+s+sd}{    }
        \PY{l+s+sd}{    Returns:}
        \PY{l+s+sd}{    params \PYZhy{}\PYZhy{} dictionary containing the weights w and bias b}
        \PY{l+s+sd}{    grads \PYZhy{}\PYZhy{} dictionary containing the gradients of the weights and bias with respect to the cost function}
        \PY{l+s+sd}{    costs \PYZhy{}\PYZhy{} list of all the costs computed during the optimization, this will be used to plot the learning curve.}
        \PY{l+s+sd}{    }
        \PY{l+s+sd}{    Tips:}
        \PY{l+s+sd}{    You basically need to write down two steps and iterate through them:}
        \PY{l+s+sd}{        1) Calculate the cost and the gradient for the current parameters. Use propagate().}
        \PY{l+s+sd}{        2) Update the parameters using gradient descent rule for w and b.}
        \PY{l+s+sd}{    \PYZdq{}\PYZdq{}\PYZdq{}}
            
            \PY{n}{costs} \PY{o}{=} \PY{p}{[}\PY{p}{]}
            
            \PY{k}{for} \PY{n}{i} \PY{o+ow}{in} \PY{n+nb}{range}\PY{p}{(}\PY{n}{num\PYZus{}iterations}\PY{p}{)}\PY{p}{:}
                
                
                \PY{c+c1}{\PYZsh{} Cost and gradient calculation (≈ 1\PYZhy{}4 lines of code)}
                \PY{c+c1}{\PYZsh{}\PYZsh{}\PYZsh{} START CODE HERE \PYZsh{}\PYZsh{}\PYZsh{} }
                \PY{n}{grads}\PY{p}{,} \PY{n}{cost} \PY{o}{=} \PY{k+kc}{None}
                \PY{c+c1}{\PYZsh{}\PYZsh{}\PYZsh{} END CODE HERE \PYZsh{}\PYZsh{}\PYZsh{}}
                
                \PY{c+c1}{\PYZsh{} Retrieve derivatives from grads}
                \PY{n}{dw} \PY{o}{=} \PY{n}{grads}\PY{p}{[}\PY{l+s+s2}{\PYZdq{}}\PY{l+s+s2}{dw}\PY{l+s+s2}{\PYZdq{}}\PY{p}{]}
                \PY{n}{db} \PY{o}{=} \PY{n}{grads}\PY{p}{[}\PY{l+s+s2}{\PYZdq{}}\PY{l+s+s2}{db}\PY{l+s+s2}{\PYZdq{}}\PY{p}{]}
                
                \PY{c+c1}{\PYZsh{} update rule (≈ 2 lines of code)}
                \PY{c+c1}{\PYZsh{}\PYZsh{}\PYZsh{} START CODE HERE \PYZsh{}\PYZsh{}\PYZsh{}}
                \PY{n}{w} \PY{o}{=} \PY{k+kc}{None}
                \PY{n}{b} \PY{o}{=} \PY{k+kc}{None}
                \PY{c+c1}{\PYZsh{}\PYZsh{}\PYZsh{} END CODE HERE \PYZsh{}\PYZsh{}\PYZsh{}}
                
                \PY{c+c1}{\PYZsh{} Record the costs}
                \PY{k}{if} \PY{n}{i} \PY{o}{\PYZpc{}} \PY{l+m+mi}{100} \PY{o}{==} \PY{l+m+mi}{0}\PY{p}{:}
                    \PY{n}{costs}\PY{o}{.}\PY{n}{append}\PY{p}{(}\PY{n}{cost}\PY{p}{)}
                
                \PY{c+c1}{\PYZsh{} Print the cost every 100 training iterations}
                \PY{k}{if} \PY{n}{print\PYZus{}cost} \PY{o+ow}{and} \PY{n}{i} \PY{o}{\PYZpc{}} \PY{l+m+mi}{100} \PY{o}{==} \PY{l+m+mi}{0}\PY{p}{:}
                    \PY{n+nb}{print} \PY{p}{(}\PY{l+s+s2}{\PYZdq{}}\PY{l+s+s2}{Cost after iteration }\PY{l+s+si}{\PYZpc{}i}\PY{l+s+s2}{: }\PY{l+s+si}{\PYZpc{}f}\PY{l+s+s2}{\PYZdq{}} \PY{o}{\PYZpc{}}\PY{p}{(}\PY{n}{i}\PY{p}{,} \PY{n}{cost}\PY{p}{)}\PY{p}{)}
            
            \PY{n}{params} \PY{o}{=} \PY{p}{\PYZob{}}\PY{l+s+s2}{\PYZdq{}}\PY{l+s+s2}{w}\PY{l+s+s2}{\PYZdq{}}\PY{p}{:} \PY{n}{w}\PY{p}{,}
                      \PY{l+s+s2}{\PYZdq{}}\PY{l+s+s2}{b}\PY{l+s+s2}{\PYZdq{}}\PY{p}{:} \PY{n}{b}\PY{p}{\PYZcb{}}
            
            \PY{n}{grads} \PY{o}{=} \PY{p}{\PYZob{}}\PY{l+s+s2}{\PYZdq{}}\PY{l+s+s2}{dw}\PY{l+s+s2}{\PYZdq{}}\PY{p}{:} \PY{n}{dw}\PY{p}{,}
                     \PY{l+s+s2}{\PYZdq{}}\PY{l+s+s2}{db}\PY{l+s+s2}{\PYZdq{}}\PY{p}{:} \PY{n}{db}\PY{p}{\PYZcb{}}
            
            \PY{k}{return} \PY{n}{params}\PY{p}{,} \PY{n}{grads}\PY{p}{,} \PY{n}{costs}
\end{Verbatim}

    \begin{Verbatim}[commandchars=\\\{\}]
{\color{incolor}In [{\color{incolor} }]:} \PY{n}{params}\PY{p}{,} \PY{n}{grads}\PY{p}{,} \PY{n}{costs} \PY{o}{=} \PY{n}{optimize}\PY{p}{(}\PY{n}{w}\PY{p}{,} \PY{n}{b}\PY{p}{,} \PY{n}{X}\PY{p}{,} \PY{n}{Y}\PY{p}{,} \PY{n}{num\PYZus{}iterations}\PY{o}{=} \PY{l+m+mi}{100}\PY{p}{,} \PY{n}{learning\PYZus{}rate} \PY{o}{=} \PY{l+m+mf}{0.009}\PY{p}{,} \PY{n}{print\PYZus{}cost} \PY{o}{=} \PY{k+kc}{False}\PY{p}{)}
        
        \PY{n+nb}{print} \PY{p}{(}\PY{l+s+s2}{\PYZdq{}}\PY{l+s+s2}{w = }\PY{l+s+s2}{\PYZdq{}} \PY{o}{+} \PY{n+nb}{str}\PY{p}{(}\PY{n}{params}\PY{p}{[}\PY{l+s+s2}{\PYZdq{}}\PY{l+s+s2}{w}\PY{l+s+s2}{\PYZdq{}}\PY{p}{]}\PY{p}{)}\PY{p}{)}
        \PY{n+nb}{print} \PY{p}{(}\PY{l+s+s2}{\PYZdq{}}\PY{l+s+s2}{b = }\PY{l+s+s2}{\PYZdq{}} \PY{o}{+} \PY{n+nb}{str}\PY{p}{(}\PY{n}{params}\PY{p}{[}\PY{l+s+s2}{\PYZdq{}}\PY{l+s+s2}{b}\PY{l+s+s2}{\PYZdq{}}\PY{p}{]}\PY{p}{)}\PY{p}{)}
        \PY{n+nb}{print} \PY{p}{(}\PY{l+s+s2}{\PYZdq{}}\PY{l+s+s2}{dw = }\PY{l+s+s2}{\PYZdq{}} \PY{o}{+} \PY{n+nb}{str}\PY{p}{(}\PY{n}{grads}\PY{p}{[}\PY{l+s+s2}{\PYZdq{}}\PY{l+s+s2}{dw}\PY{l+s+s2}{\PYZdq{}}\PY{p}{]}\PY{p}{)}\PY{p}{)}
        \PY{n+nb}{print} \PY{p}{(}\PY{l+s+s2}{\PYZdq{}}\PY{l+s+s2}{db = }\PY{l+s+s2}{\PYZdq{}} \PY{o}{+} \PY{n+nb}{str}\PY{p}{(}\PY{n}{grads}\PY{p}{[}\PY{l+s+s2}{\PYZdq{}}\PY{l+s+s2}{db}\PY{l+s+s2}{\PYZdq{}}\PY{p}{]}\PY{p}{)}\PY{p}{)}
\end{Verbatim}

    \textbf{Expected Output}:

\textbf{w}

{[}{[} 0.19033591{]}{[} 0.12259159{]}{]}

\begin{verbatim}
<tr>
   <td> **b** </td>
   <td> 1.92535983008 </td>
</tr>
<tr>
   <td> **dw** </td>
   <td> [[ 0.67752042]
\end{verbatim}

{[} 1.41625495{]}{]}

\textbf{db}

0.219194504541

    \textbf{Exercise:} The previous function will output the learned w and
b. We are able to use w and b to predict the labels for a dataset X.
Implement the \texttt{predict()} function. There are two steps to
computing predictions:

\begin{enumerate}
\def\labelenumi{\arabic{enumi}.}
\item
  Calculate $\hat{Y} = A = \sigma(w^T X + b)$
\item
  Convert the entries of a into 0 (if activation \textless{}= 0.5) or 1
  (if activation \textgreater{} 0.5), stores the predictions in a vector
  \texttt{Y\_prediction}. If you wish, you can use an
  \texttt{if}/\texttt{else} statement in a \texttt{for} loop (though
  there is also a way to vectorize this).
\end{enumerate}

    \begin{Verbatim}[commandchars=\\\{\}]
{\color{incolor}In [{\color{incolor} }]:} \PY{c+c1}{\PYZsh{} GRADED FUNCTION: predict}
        
        \PY{k}{def} \PY{n+nf}{predict}\PY{p}{(}\PY{n}{w}\PY{p}{,} \PY{n}{b}\PY{p}{,} \PY{n}{X}\PY{p}{)}\PY{p}{:}
            \PY{l+s+sd}{\PYZsq{}\PYZsq{}\PYZsq{}}
        \PY{l+s+sd}{    Predict whether the label is 0 or 1 using learned logistic regression parameters (w, b)}
        \PY{l+s+sd}{    }
        \PY{l+s+sd}{    Arguments:}
        \PY{l+s+sd}{    w \PYZhy{}\PYZhy{} weights, a numpy array of size (num\PYZus{}px * num\PYZus{}px * 3, 1)}
        \PY{l+s+sd}{    b \PYZhy{}\PYZhy{} bias, a scalar}
        \PY{l+s+sd}{    X \PYZhy{}\PYZhy{} data of size (num\PYZus{}px * num\PYZus{}px * 3, number of examples)}
        \PY{l+s+sd}{    }
        \PY{l+s+sd}{    Returns:}
        \PY{l+s+sd}{    Y\PYZus{}prediction \PYZhy{}\PYZhy{} a numpy array (vector) containing all predictions (0/1) for the examples in X}
        \PY{l+s+sd}{    \PYZsq{}\PYZsq{}\PYZsq{}}
            
            \PY{n}{m} \PY{o}{=} \PY{n}{X}\PY{o}{.}\PY{n}{shape}\PY{p}{[}\PY{l+m+mi}{1}\PY{p}{]}
            \PY{n}{Y\PYZus{}prediction} \PY{o}{=} \PY{n}{np}\PY{o}{.}\PY{n}{zeros}\PY{p}{(}\PY{p}{(}\PY{l+m+mi}{1}\PY{p}{,}\PY{n}{m}\PY{p}{)}\PY{p}{)}
            \PY{n}{w} \PY{o}{=} \PY{n}{w}\PY{o}{.}\PY{n}{reshape}\PY{p}{(}\PY{n}{X}\PY{o}{.}\PY{n}{shape}\PY{p}{[}\PY{l+m+mi}{0}\PY{p}{]}\PY{p}{,} \PY{l+m+mi}{1}\PY{p}{)}
            
            \PY{c+c1}{\PYZsh{} Compute vector \PYZdq{}A\PYZdq{} predicting the probabilities of a cat being present in the picture}
            \PY{c+c1}{\PYZsh{}\PYZsh{}\PYZsh{} START CODE HERE \PYZsh{}\PYZsh{}\PYZsh{} (≈ 1 line of code)}
            \PY{n}{A} \PY{o}{=} \PY{k+kc}{None}
            \PY{c+c1}{\PYZsh{}\PYZsh{}\PYZsh{} END CODE HERE \PYZsh{}\PYZsh{}\PYZsh{}}
            
            \PY{k}{for} \PY{n}{i} \PY{o+ow}{in} \PY{n+nb}{range}\PY{p}{(}\PY{n}{A}\PY{o}{.}\PY{n}{shape}\PY{p}{[}\PY{l+m+mi}{1}\PY{p}{]}\PY{p}{)}\PY{p}{:}
                
                \PY{c+c1}{\PYZsh{} Convert probabilities A[0,i] to actual predictions p[0,i]}
                \PY{c+c1}{\PYZsh{}\PYZsh{}\PYZsh{} START CODE HERE \PYZsh{}\PYZsh{}\PYZsh{} (≈ 4 lines of code)}
                \PY{k}{pass}
                \PY{c+c1}{\PYZsh{}\PYZsh{}\PYZsh{} END CODE HERE \PYZsh{}\PYZsh{}\PYZsh{}}
            
            \PY{k}{assert}\PY{p}{(}\PY{n}{Y\PYZus{}prediction}\PY{o}{.}\PY{n}{shape} \PY{o}{==} \PY{p}{(}\PY{l+m+mi}{1}\PY{p}{,} \PY{n}{m}\PY{p}{)}\PY{p}{)}
            
            \PY{k}{return} \PY{n}{Y\PYZus{}prediction}
\end{Verbatim}

    \begin{Verbatim}[commandchars=\\\{\}]
{\color{incolor}In [{\color{incolor} }]:} \PY{n}{w} \PY{o}{=} \PY{n}{np}\PY{o}{.}\PY{n}{array}\PY{p}{(}\PY{p}{[}\PY{p}{[}\PY{l+m+mf}{0.1124579}\PY{p}{]}\PY{p}{,}\PY{p}{[}\PY{l+m+mf}{0.23106775}\PY{p}{]}\PY{p}{]}\PY{p}{)}
        \PY{n}{b} \PY{o}{=} \PY{o}{\PYZhy{}}\PY{l+m+mf}{0.3}
        \PY{n}{X} \PY{o}{=} \PY{n}{np}\PY{o}{.}\PY{n}{array}\PY{p}{(}\PY{p}{[}\PY{p}{[}\PY{l+m+mf}{1.}\PY{p}{,}\PY{o}{\PYZhy{}}\PY{l+m+mf}{1.1}\PY{p}{,}\PY{o}{\PYZhy{}}\PY{l+m+mf}{3.2}\PY{p}{]}\PY{p}{,}\PY{p}{[}\PY{l+m+mf}{1.2}\PY{p}{,}\PY{l+m+mf}{2.}\PY{p}{,}\PY{l+m+mf}{0.1}\PY{p}{]}\PY{p}{]}\PY{p}{)}
        \PY{n+nb}{print} \PY{p}{(}\PY{l+s+s2}{\PYZdq{}}\PY{l+s+s2}{predictions = }\PY{l+s+s2}{\PYZdq{}} \PY{o}{+} \PY{n+nb}{str}\PY{p}{(}\PY{n}{predict}\PY{p}{(}\PY{n}{w}\PY{p}{,} \PY{n}{b}\PY{p}{,} \PY{n}{X}\PY{p}{)}\PY{p}{)}\PY{p}{)}
\end{Verbatim}

    \textbf{Expected Output}:

\textbf{predictions}

{[}{[} 1. 1. 0.{]}{]}

     \textbf{What to remember:} You've implemented several functions that: -
Initialize (w,b) - Optimize the loss iteratively to learn parameters
(w,b): - computing the cost and its gradient - updating the parameters
using gradient descent - Use the learned (w,b) to predict the labels for
a given set of examples

    \subsection{5 - Merge all functions into a
model}\label{merge-all-functions-into-a-model}

You will now see how the overall model is structured by putting together
all the building blocks (functions implemented in the previous parts)
together, in the right order.

\textbf{Exercise:} Implement the model function. Use the following
notation: - Y\_prediction\_test for your predictions on the test set -
Y\_prediction\_train for your predictions on the train set - w, costs,
grads for the outputs of optimize()

    \begin{Verbatim}[commandchars=\\\{\}]
{\color{incolor}In [{\color{incolor} }]:} \PY{c+c1}{\PYZsh{} GRADED FUNCTION: model}
        
        \PY{k}{def} \PY{n+nf}{model}\PY{p}{(}\PY{n}{X\PYZus{}train}\PY{p}{,} \PY{n}{Y\PYZus{}train}\PY{p}{,} \PY{n}{X\PYZus{}test}\PY{p}{,} \PY{n}{Y\PYZus{}test}\PY{p}{,} \PY{n}{num\PYZus{}iterations} \PY{o}{=} \PY{l+m+mi}{2000}\PY{p}{,} \PY{n}{learning\PYZus{}rate} \PY{o}{=} \PY{l+m+mf}{0.5}\PY{p}{,} \PY{n}{print\PYZus{}cost} \PY{o}{=} \PY{k+kc}{False}\PY{p}{)}\PY{p}{:}
            \PY{l+s+sd}{\PYZdq{}\PYZdq{}\PYZdq{}}
        \PY{l+s+sd}{    Builds the logistic regression model by calling the function you\PYZsq{}ve implemented previously}
        \PY{l+s+sd}{    }
        \PY{l+s+sd}{    Arguments:}
        \PY{l+s+sd}{    X\PYZus{}train \PYZhy{}\PYZhy{} training set represented by a numpy array of shape (num\PYZus{}px * num\PYZus{}px * 3, m\PYZus{}train)}
        \PY{l+s+sd}{    Y\PYZus{}train \PYZhy{}\PYZhy{} training labels represented by a numpy array (vector) of shape (1, m\PYZus{}train)}
        \PY{l+s+sd}{    X\PYZus{}test \PYZhy{}\PYZhy{} test set represented by a numpy array of shape (num\PYZus{}px * num\PYZus{}px * 3, m\PYZus{}test)}
        \PY{l+s+sd}{    Y\PYZus{}test \PYZhy{}\PYZhy{} test labels represented by a numpy array (vector) of shape (1, m\PYZus{}test)}
        \PY{l+s+sd}{    num\PYZus{}iterations \PYZhy{}\PYZhy{} hyperparameter representing the number of iterations to optimize the parameters}
        \PY{l+s+sd}{    learning\PYZus{}rate \PYZhy{}\PYZhy{} hyperparameter representing the learning rate used in the update rule of optimize()}
        \PY{l+s+sd}{    print\PYZus{}cost \PYZhy{}\PYZhy{} Set to true to print the cost every 100 iterations}
        \PY{l+s+sd}{    }
        \PY{l+s+sd}{    Returns:}
        \PY{l+s+sd}{    d \PYZhy{}\PYZhy{} dictionary containing information about the model.}
        \PY{l+s+sd}{    \PYZdq{}\PYZdq{}\PYZdq{}}
            
            \PY{c+c1}{\PYZsh{}\PYZsh{}\PYZsh{} START CODE HERE \PYZsh{}\PYZsh{}\PYZsh{}}
            
            \PY{c+c1}{\PYZsh{} initialize parameters with zeros (≈ 1 line of code)}
            \PY{n}{w}\PY{p}{,} \PY{n}{b} \PY{o}{=} \PY{k+kc}{None}
        
            \PY{c+c1}{\PYZsh{} Gradient descent (≈ 1 line of code)}
            \PY{n}{parameters}\PY{p}{,} \PY{n}{grads}\PY{p}{,} \PY{n}{costs} \PY{o}{=} \PY{k+kc}{None}
            
            \PY{c+c1}{\PYZsh{} Retrieve parameters w and b from dictionary \PYZdq{}parameters\PYZdq{}}
            \PY{n}{w} \PY{o}{=} \PY{n}{parameters}\PY{p}{[}\PY{l+s+s2}{\PYZdq{}}\PY{l+s+s2}{w}\PY{l+s+s2}{\PYZdq{}}\PY{p}{]}
            \PY{n}{b} \PY{o}{=} \PY{n}{parameters}\PY{p}{[}\PY{l+s+s2}{\PYZdq{}}\PY{l+s+s2}{b}\PY{l+s+s2}{\PYZdq{}}\PY{p}{]}
            
            \PY{c+c1}{\PYZsh{} Predict test/train set examples (≈ 2 lines of code)}
            \PY{n}{Y\PYZus{}prediction\PYZus{}test} \PY{o}{=} \PY{k+kc}{None}
            \PY{n}{Y\PYZus{}prediction\PYZus{}train} \PY{o}{=} \PY{k+kc}{None}
        
            \PY{c+c1}{\PYZsh{}\PYZsh{}\PYZsh{} END CODE HERE \PYZsh{}\PYZsh{}\PYZsh{}}
        
            \PY{c+c1}{\PYZsh{} Print train/test Errors}
            \PY{n+nb}{print}\PY{p}{(}\PY{l+s+s2}{\PYZdq{}}\PY{l+s+s2}{train accuracy: }\PY{l+s+si}{\PYZob{}\PYZcb{}}\PY{l+s+s2}{ }\PY{l+s+s2}{\PYZpc{}}\PY{l+s+s2}{\PYZdq{}}\PY{o}{.}\PY{n}{format}\PY{p}{(}\PY{l+m+mi}{100} \PY{o}{\PYZhy{}} \PY{n}{np}\PY{o}{.}\PY{n}{mean}\PY{p}{(}\PY{n}{np}\PY{o}{.}\PY{n}{abs}\PY{p}{(}\PY{n}{Y\PYZus{}prediction\PYZus{}train} \PY{o}{\PYZhy{}} \PY{n}{Y\PYZus{}train}\PY{p}{)}\PY{p}{)} \PY{o}{*} \PY{l+m+mi}{100}\PY{p}{)}\PY{p}{)}
            \PY{n+nb}{print}\PY{p}{(}\PY{l+s+s2}{\PYZdq{}}\PY{l+s+s2}{test accuracy: }\PY{l+s+si}{\PYZob{}\PYZcb{}}\PY{l+s+s2}{ }\PY{l+s+s2}{\PYZpc{}}\PY{l+s+s2}{\PYZdq{}}\PY{o}{.}\PY{n}{format}\PY{p}{(}\PY{l+m+mi}{100} \PY{o}{\PYZhy{}} \PY{n}{np}\PY{o}{.}\PY{n}{mean}\PY{p}{(}\PY{n}{np}\PY{o}{.}\PY{n}{abs}\PY{p}{(}\PY{n}{Y\PYZus{}prediction\PYZus{}test} \PY{o}{\PYZhy{}} \PY{n}{Y\PYZus{}test}\PY{p}{)}\PY{p}{)} \PY{o}{*} \PY{l+m+mi}{100}\PY{p}{)}\PY{p}{)}
        
            
            \PY{n}{d} \PY{o}{=} \PY{p}{\PYZob{}}\PY{l+s+s2}{\PYZdq{}}\PY{l+s+s2}{costs}\PY{l+s+s2}{\PYZdq{}}\PY{p}{:} \PY{n}{costs}\PY{p}{,}
                 \PY{l+s+s2}{\PYZdq{}}\PY{l+s+s2}{Y\PYZus{}prediction\PYZus{}test}\PY{l+s+s2}{\PYZdq{}}\PY{p}{:} \PY{n}{Y\PYZus{}prediction\PYZus{}test}\PY{p}{,} 
                 \PY{l+s+s2}{\PYZdq{}}\PY{l+s+s2}{Y\PYZus{}prediction\PYZus{}train}\PY{l+s+s2}{\PYZdq{}} \PY{p}{:} \PY{n}{Y\PYZus{}prediction\PYZus{}train}\PY{p}{,} 
                 \PY{l+s+s2}{\PYZdq{}}\PY{l+s+s2}{w}\PY{l+s+s2}{\PYZdq{}} \PY{p}{:} \PY{n}{w}\PY{p}{,} 
                 \PY{l+s+s2}{\PYZdq{}}\PY{l+s+s2}{b}\PY{l+s+s2}{\PYZdq{}} \PY{p}{:} \PY{n}{b}\PY{p}{,}
                 \PY{l+s+s2}{\PYZdq{}}\PY{l+s+s2}{learning\PYZus{}rate}\PY{l+s+s2}{\PYZdq{}} \PY{p}{:} \PY{n}{learning\PYZus{}rate}\PY{p}{,}
                 \PY{l+s+s2}{\PYZdq{}}\PY{l+s+s2}{num\PYZus{}iterations}\PY{l+s+s2}{\PYZdq{}}\PY{p}{:} \PY{n}{num\PYZus{}iterations}\PY{p}{\PYZcb{}}
            
            \PY{k}{return} \PY{n}{d}
\end{Verbatim}

    Run the following cell to train your model.

    \begin{Verbatim}[commandchars=\\\{\}]
{\color{incolor}In [{\color{incolor} }]:} \PY{n}{d} \PY{o}{=} \PY{n}{model}\PY{p}{(}\PY{n}{train\PYZus{}set\PYZus{}x}\PY{p}{,} \PY{n}{train\PYZus{}set\PYZus{}y}\PY{p}{,} \PY{n}{test\PYZus{}set\PYZus{}x}\PY{p}{,} \PY{n}{test\PYZus{}set\PYZus{}y}\PY{p}{,} \PY{n}{num\PYZus{}iterations} \PY{o}{=} \PY{l+m+mi}{2000}\PY{p}{,} \PY{n}{learning\PYZus{}rate} \PY{o}{=} \PY{l+m+mf}{0.005}\PY{p}{,} \PY{n}{print\PYZus{}cost} \PY{o}{=} \PY{k+kc}{True}\PY{p}{)}
\end{Verbatim}

    \textbf{Expected Output}:

\begin{verbatim}
<tr>
    <td> **Cost after iteration 0 **  </td> 
    <td> 0.693147 </td>
</tr>
  <tr>
    <td> <center> $\vdots$ </center> </td> 
    <td> <center> $\vdots$ </center> </td> 
</tr>  
<tr>
    <td> **Train Accuracy**  </td> 
    <td> 99.04306220095694 % </td>
</tr>

<tr>
    <td>**Test Accuracy** </td> 
    <td> 70.0 % </td>
</tr>
\end{verbatim}

    \textbf{Comment}: Training accuracy is close to 100\%. This is a good
sanity check: your model is working and has high enough capacity to fit
the training data. Test accuracy is 68\%. It is actually not bad for
this simple model, given the small dataset we used and that logistic
regression is a linear classifier. But no worries, you'll build an even
better classifier next week!

Also, you see that the model is clearly overfitting the training data.
Later in this specialization you will learn how to reduce overfitting,
for example by using regularization. Using the code below (and changing
the \texttt{index} variable) you can look at predictions on pictures of
the test set.

    \begin{Verbatim}[commandchars=\\\{\}]
{\color{incolor}In [{\color{incolor} }]:} \PY{c+c1}{\PYZsh{} Example of a picture that was wrongly classified.}
        \PY{n}{index} \PY{o}{=} \PY{l+m+mi}{1}
        \PY{n}{plt}\PY{o}{.}\PY{n}{imshow}\PY{p}{(}\PY{n}{test\PYZus{}set\PYZus{}x}\PY{p}{[}\PY{p}{:}\PY{p}{,}\PY{n}{index}\PY{p}{]}\PY{o}{.}\PY{n}{reshape}\PY{p}{(}\PY{p}{(}\PY{n}{num\PYZus{}px}\PY{p}{,} \PY{n}{num\PYZus{}px}\PY{p}{,} \PY{l+m+mi}{3}\PY{p}{)}\PY{p}{)}\PY{p}{)}
        \PY{n+nb}{print} \PY{p}{(}\PY{l+s+s2}{\PYZdq{}}\PY{l+s+s2}{y = }\PY{l+s+s2}{\PYZdq{}} \PY{o}{+} \PY{n+nb}{str}\PY{p}{(}\PY{n}{test\PYZus{}set\PYZus{}y}\PY{p}{[}\PY{l+m+mi}{0}\PY{p}{,}\PY{n}{index}\PY{p}{]}\PY{p}{)} \PY{o}{+} \PY{l+s+s2}{\PYZdq{}}\PY{l+s+s2}{, you predicted that it is a }\PY{l+s+se}{\PYZbs{}\PYZdq{}}\PY{l+s+s2}{\PYZdq{}} \PY{o}{+} \PY{n}{classes}\PY{p}{[}\PY{n}{d}\PY{p}{[}\PY{l+s+s2}{\PYZdq{}}\PY{l+s+s2}{Y\PYZus{}prediction\PYZus{}test}\PY{l+s+s2}{\PYZdq{}}\PY{p}{]}\PY{p}{[}\PY{l+m+mi}{0}\PY{p}{,}\PY{n}{index}\PY{p}{]}\PY{p}{]}\PY{o}{.}\PY{n}{decode}\PY{p}{(}\PY{l+s+s2}{\PYZdq{}}\PY{l+s+s2}{utf\PYZhy{}8}\PY{l+s+s2}{\PYZdq{}}\PY{p}{)} \PY{o}{+}  \PY{l+s+s2}{\PYZdq{}}\PY{l+s+se}{\PYZbs{}\PYZdq{}}\PY{l+s+s2}{ picture.}\PY{l+s+s2}{\PYZdq{}}\PY{p}{)}
\end{Verbatim}

    Let's also plot the cost function and the gradients.

    \begin{Verbatim}[commandchars=\\\{\}]
{\color{incolor}In [{\color{incolor} }]:} \PY{c+c1}{\PYZsh{} Plot learning curve (with costs)}
        \PY{n}{costs} \PY{o}{=} \PY{n}{np}\PY{o}{.}\PY{n}{squeeze}\PY{p}{(}\PY{n}{d}\PY{p}{[}\PY{l+s+s1}{\PYZsq{}}\PY{l+s+s1}{costs}\PY{l+s+s1}{\PYZsq{}}\PY{p}{]}\PY{p}{)}
        \PY{n}{plt}\PY{o}{.}\PY{n}{plot}\PY{p}{(}\PY{n}{costs}\PY{p}{)}
        \PY{n}{plt}\PY{o}{.}\PY{n}{ylabel}\PY{p}{(}\PY{l+s+s1}{\PYZsq{}}\PY{l+s+s1}{cost}\PY{l+s+s1}{\PYZsq{}}\PY{p}{)}
        \PY{n}{plt}\PY{o}{.}\PY{n}{xlabel}\PY{p}{(}\PY{l+s+s1}{\PYZsq{}}\PY{l+s+s1}{iterations (per hundreds)}\PY{l+s+s1}{\PYZsq{}}\PY{p}{)}
        \PY{n}{plt}\PY{o}{.}\PY{n}{title}\PY{p}{(}\PY{l+s+s2}{\PYZdq{}}\PY{l+s+s2}{Learning rate =}\PY{l+s+s2}{\PYZdq{}} \PY{o}{+} \PY{n+nb}{str}\PY{p}{(}\PY{n}{d}\PY{p}{[}\PY{l+s+s2}{\PYZdq{}}\PY{l+s+s2}{learning\PYZus{}rate}\PY{l+s+s2}{\PYZdq{}}\PY{p}{]}\PY{p}{)}\PY{p}{)}
        \PY{n}{plt}\PY{o}{.}\PY{n}{show}\PY{p}{(}\PY{p}{)}
\end{Verbatim}

    \textbf{Interpretation}: You can see the cost decreasing. It shows that
the parameters are being learned. However, you see that you could train
the model even more on the training set. Try to increase the number of
iterations in the cell above and rerun the cells. You might see that the
training set accuracy goes up, but the test set accuracy goes down. This
is called overfitting.

    \subsection{6 - Further analysis (optional/ungraded
exercise)}\label{further-analysis-optionalungraded-exercise}

Congratulations on building your first image classification model. Let's
analyze it further, and examine possible choices for the learning rate
$\alpha$.

    \paragraph{Choice of learning rate}\label{choice-of-learning-rate}

\textbf{Reminder}: In order for Gradient Descent to work you must choose
the learning rate wisely. The learning rate $\alpha$ determines how
rapidly we update the parameters. If the learning rate is too large we
may ``overshoot'' the optimal value. Similarly, if it is too small we
will need too many iterations to converge to the best values. That's why
it is crucial to use a well-tuned learning rate.

Let's compare the learning curve of our model with several choices of
learning rates. Run the cell below. This should take about 1 minute.
Feel free also to try different values than the three we have
initialized the \texttt{learning\_rates} variable to contain, and see
what happens.

    \begin{Verbatim}[commandchars=\\\{\}]
{\color{incolor}In [{\color{incolor} }]:} \PY{n}{learning\PYZus{}rates} \PY{o}{=} \PY{p}{[}\PY{l+m+mf}{0.01}\PY{p}{,} \PY{l+m+mf}{0.001}\PY{p}{,} \PY{l+m+mf}{0.0001}\PY{p}{]}
        \PY{n}{models} \PY{o}{=} \PY{p}{\PYZob{}}\PY{p}{\PYZcb{}}
        \PY{k}{for} \PY{n}{i} \PY{o+ow}{in} \PY{n}{learning\PYZus{}rates}\PY{p}{:}
            \PY{n+nb}{print} \PY{p}{(}\PY{l+s+s2}{\PYZdq{}}\PY{l+s+s2}{learning rate is: }\PY{l+s+s2}{\PYZdq{}} \PY{o}{+} \PY{n+nb}{str}\PY{p}{(}\PY{n}{i}\PY{p}{)}\PY{p}{)}
            \PY{n}{models}\PY{p}{[}\PY{n+nb}{str}\PY{p}{(}\PY{n}{i}\PY{p}{)}\PY{p}{]} \PY{o}{=} \PY{n}{model}\PY{p}{(}\PY{n}{train\PYZus{}set\PYZus{}x}\PY{p}{,} \PY{n}{train\PYZus{}set\PYZus{}y}\PY{p}{,} \PY{n}{test\PYZus{}set\PYZus{}x}\PY{p}{,} \PY{n}{test\PYZus{}set\PYZus{}y}\PY{p}{,} \PY{n}{num\PYZus{}iterations} \PY{o}{=} \PY{l+m+mi}{1500}\PY{p}{,} \PY{n}{learning\PYZus{}rate} \PY{o}{=} \PY{n}{i}\PY{p}{,} \PY{n}{print\PYZus{}cost} \PY{o}{=} \PY{k+kc}{False}\PY{p}{)}
            \PY{n+nb}{print} \PY{p}{(}\PY{l+s+s1}{\PYZsq{}}\PY{l+s+se}{\PYZbs{}n}\PY{l+s+s1}{\PYZsq{}} \PY{o}{+} \PY{l+s+s2}{\PYZdq{}}\PY{l+s+s2}{\PYZhy{}\PYZhy{}\PYZhy{}\PYZhy{}\PYZhy{}\PYZhy{}\PYZhy{}\PYZhy{}\PYZhy{}\PYZhy{}\PYZhy{}\PYZhy{}\PYZhy{}\PYZhy{}\PYZhy{}\PYZhy{}\PYZhy{}\PYZhy{}\PYZhy{}\PYZhy{}\PYZhy{}\PYZhy{}\PYZhy{}\PYZhy{}\PYZhy{}\PYZhy{}\PYZhy{}\PYZhy{}\PYZhy{}\PYZhy{}\PYZhy{}\PYZhy{}\PYZhy{}\PYZhy{}\PYZhy{}\PYZhy{}\PYZhy{}\PYZhy{}\PYZhy{}\PYZhy{}\PYZhy{}\PYZhy{}\PYZhy{}\PYZhy{}\PYZhy{}\PYZhy{}\PYZhy{}\PYZhy{}\PYZhy{}\PYZhy{}\PYZhy{}\PYZhy{}\PYZhy{}\PYZhy{}\PYZhy{}}\PY{l+s+s2}{\PYZdq{}} \PY{o}{+} \PY{l+s+s1}{\PYZsq{}}\PY{l+s+se}{\PYZbs{}n}\PY{l+s+s1}{\PYZsq{}}\PY{p}{)}
        
        \PY{k}{for} \PY{n}{i} \PY{o+ow}{in} \PY{n}{learning\PYZus{}rates}\PY{p}{:}
            \PY{n}{plt}\PY{o}{.}\PY{n}{plot}\PY{p}{(}\PY{n}{np}\PY{o}{.}\PY{n}{squeeze}\PY{p}{(}\PY{n}{models}\PY{p}{[}\PY{n+nb}{str}\PY{p}{(}\PY{n}{i}\PY{p}{)}\PY{p}{]}\PY{p}{[}\PY{l+s+s2}{\PYZdq{}}\PY{l+s+s2}{costs}\PY{l+s+s2}{\PYZdq{}}\PY{p}{]}\PY{p}{)}\PY{p}{,} \PY{n}{label}\PY{o}{=} \PY{n+nb}{str}\PY{p}{(}\PY{n}{models}\PY{p}{[}\PY{n+nb}{str}\PY{p}{(}\PY{n}{i}\PY{p}{)}\PY{p}{]}\PY{p}{[}\PY{l+s+s2}{\PYZdq{}}\PY{l+s+s2}{learning\PYZus{}rate}\PY{l+s+s2}{\PYZdq{}}\PY{p}{]}\PY{p}{)}\PY{p}{)}
        
        \PY{n}{plt}\PY{o}{.}\PY{n}{ylabel}\PY{p}{(}\PY{l+s+s1}{\PYZsq{}}\PY{l+s+s1}{cost}\PY{l+s+s1}{\PYZsq{}}\PY{p}{)}
        \PY{n}{plt}\PY{o}{.}\PY{n}{xlabel}\PY{p}{(}\PY{l+s+s1}{\PYZsq{}}\PY{l+s+s1}{iterations (hundreds)}\PY{l+s+s1}{\PYZsq{}}\PY{p}{)}
        
        \PY{n}{legend} \PY{o}{=} \PY{n}{plt}\PY{o}{.}\PY{n}{legend}\PY{p}{(}\PY{n}{loc}\PY{o}{=}\PY{l+s+s1}{\PYZsq{}}\PY{l+s+s1}{upper center}\PY{l+s+s1}{\PYZsq{}}\PY{p}{,} \PY{n}{shadow}\PY{o}{=}\PY{k+kc}{True}\PY{p}{)}
        \PY{n}{frame} \PY{o}{=} \PY{n}{legend}\PY{o}{.}\PY{n}{get\PYZus{}frame}\PY{p}{(}\PY{p}{)}
        \PY{n}{frame}\PY{o}{.}\PY{n}{set\PYZus{}facecolor}\PY{p}{(}\PY{l+s+s1}{\PYZsq{}}\PY{l+s+s1}{0.90}\PY{l+s+s1}{\PYZsq{}}\PY{p}{)}
        \PY{n}{plt}\PY{o}{.}\PY{n}{show}\PY{p}{(}\PY{p}{)}
\end{Verbatim}

    \textbf{Interpretation}: - Different learning rates give different costs
and thus different predictions results. - If the learning rate is too
large (0.01), the cost may oscillate up and down. It may even diverge
(though in this example, using 0.01 still eventually ends up at a good
value for the cost). - A lower cost doesn't mean a better model. You
have to check if there is possibly overfitting. It happens when the
training accuracy is a lot higher than the test accuracy. - In deep
learning, we usually recommend that you: - Choose the learning rate that
better minimizes the cost function. - If your model overfits, use other
techniques to reduce overfitting. (We'll talk about this in later
videos.)

    \subsection{7 - Test with your own image (optional/ungraded
exercise)}\label{test-with-your-own-image-optionalungraded-exercise}

Congratulations on finishing this assignment. You can use your own image
and see the output of your model. To do that: 1. Click on ``File'' in
the upper bar of this notebook, then click ``Open'' to go on your
Coursera Hub. 2. Add your image to this Jupyter Notebook's directory, in
the ``images'' folder 3. Change your image's name in the following code
4. Run the code and check if the algorithm is right (1 = cat, 0 =
non-cat)!

    \begin{Verbatim}[commandchars=\\\{\}]
{\color{incolor}In [{\color{incolor} }]:} \PY{c+c1}{\PYZsh{}\PYZsh{} START CODE HERE \PYZsh{}\PYZsh{} (PUT YOUR IMAGE NAME) }
        \PY{n}{my\PYZus{}image} \PY{o}{=} \PY{l+s+s2}{\PYZdq{}}\PY{l+s+s2}{my\PYZus{}image.jpg}\PY{l+s+s2}{\PYZdq{}}   \PY{c+c1}{\PYZsh{} change this to the name of your image file }
        \PY{c+c1}{\PYZsh{}\PYZsh{} END CODE HERE \PYZsh{}\PYZsh{}}
        
        \PY{c+c1}{\PYZsh{} We preprocess the image to fit your algorithm.}
        \PY{n}{fname} \PY{o}{=} \PY{l+s+s2}{\PYZdq{}}\PY{l+s+s2}{images/}\PY{l+s+s2}{\PYZdq{}} \PY{o}{+} \PY{n}{my\PYZus{}image}
        \PY{n}{image} \PY{o}{=} \PY{n}{np}\PY{o}{.}\PY{n}{array}\PY{p}{(}\PY{n}{ndimage}\PY{o}{.}\PY{n}{imread}\PY{p}{(}\PY{n}{fname}\PY{p}{,} \PY{n}{flatten}\PY{o}{=}\PY{k+kc}{False}\PY{p}{)}\PY{p}{)}
        \PY{n}{image} \PY{o}{=} \PY{n}{image}\PY{o}{/}\PY{l+m+mf}{255.}
        \PY{n}{my\PYZus{}image} \PY{o}{=} \PY{n}{scipy}\PY{o}{.}\PY{n}{misc}\PY{o}{.}\PY{n}{imresize}\PY{p}{(}\PY{n}{image}\PY{p}{,} \PY{n}{size}\PY{o}{=}\PY{p}{(}\PY{n}{num\PYZus{}px}\PY{p}{,}\PY{n}{num\PYZus{}px}\PY{p}{)}\PY{p}{)}\PY{o}{.}\PY{n}{reshape}\PY{p}{(}\PY{p}{(}\PY{l+m+mi}{1}\PY{p}{,} \PY{n}{num\PYZus{}px}\PY{o}{*}\PY{n}{num\PYZus{}px}\PY{o}{*}\PY{l+m+mi}{3}\PY{p}{)}\PY{p}{)}\PY{o}{.}\PY{n}{T}
        \PY{n}{my\PYZus{}predicted\PYZus{}image} \PY{o}{=} \PY{n}{predict}\PY{p}{(}\PY{n}{d}\PY{p}{[}\PY{l+s+s2}{\PYZdq{}}\PY{l+s+s2}{w}\PY{l+s+s2}{\PYZdq{}}\PY{p}{]}\PY{p}{,} \PY{n}{d}\PY{p}{[}\PY{l+s+s2}{\PYZdq{}}\PY{l+s+s2}{b}\PY{l+s+s2}{\PYZdq{}}\PY{p}{]}\PY{p}{,} \PY{n}{my\PYZus{}image}\PY{p}{)}
        
        \PY{n}{plt}\PY{o}{.}\PY{n}{imshow}\PY{p}{(}\PY{n}{image}\PY{p}{)}
        \PY{n+nb}{print}\PY{p}{(}\PY{l+s+s2}{\PYZdq{}}\PY{l+s+s2}{y = }\PY{l+s+s2}{\PYZdq{}} \PY{o}{+} \PY{n+nb}{str}\PY{p}{(}\PY{n}{np}\PY{o}{.}\PY{n}{squeeze}\PY{p}{(}\PY{n}{my\PYZus{}predicted\PYZus{}image}\PY{p}{)}\PY{p}{)} \PY{o}{+} \PY{l+s+s2}{\PYZdq{}}\PY{l+s+s2}{, your algorithm predicts a }\PY{l+s+se}{\PYZbs{}\PYZdq{}}\PY{l+s+s2}{\PYZdq{}} \PY{o}{+} \PY{n}{classes}\PY{p}{[}\PY{n+nb}{int}\PY{p}{(}\PY{n}{np}\PY{o}{.}\PY{n}{squeeze}\PY{p}{(}\PY{n}{my\PYZus{}predicted\PYZus{}image}\PY{p}{)}\PY{p}{)}\PY{p}{,}\PY{p}{]}\PY{o}{.}\PY{n}{decode}\PY{p}{(}\PY{l+s+s2}{\PYZdq{}}\PY{l+s+s2}{utf\PYZhy{}8}\PY{l+s+s2}{\PYZdq{}}\PY{p}{)} \PY{o}{+}  \PY{l+s+s2}{\PYZdq{}}\PY{l+s+se}{\PYZbs{}\PYZdq{}}\PY{l+s+s2}{ picture.}\PY{l+s+s2}{\PYZdq{}}\PY{p}{)}
\end{Verbatim}

     \textbf{What to remember from this assignment:} 1. Preprocessing the
dataset is important. 2. You implemented each function separately:
initialize(), propagate(), optimize(). Then you built a model(). 3.
Tuning the learning rate (which is an example of a ``hyperparameter'')
can make a big difference to the algorithm. You will see more examples
of this later in this course!

    Finally, if you'd like, we invite you to try different things on this
Notebook. Make sure you submit before trying anything. Once you submit,
things you can play with include: - Play with the learning rate and the
number of iterations - Try different initialization methods and compare
the results - Test other preprocessings (center the data, or divide each
row by its standard deviation)

    Bibliography: -
http://www.wildml.com/2015/09/implementing-a-neural-network-from-scratch/
-
https://stats.stackexchange.com/questions/211436/why-do-we-normalize-images-by-subtracting-the-datasets-image-mean-and-not-the-c


    % Add a bibliography block to the postdoc
    
    
    
    \end{document}
